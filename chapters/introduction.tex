\chapter{Introduction.}

\section{Grammar.}

\textsc{Grammar} is the Art of rightly expressing our thoughts by Words.

Grammar in general, or Universal Grammar, explains the principles, which
are common to all languages.

The Grammar of any particular Language, as the English Grammar, applies
those common principles to that particular language, according to the
established usage and custom of it.

Grammar treats of Sentences; and of the several parts, of which they are
compounded.

Sentences consist of Words; Words, of one or more Syllables; Syllables,
of one or more Letters.

So that Letters, Syllables, Words, and Sentences, make up the whole
subject of Grammar.

\section{Letters.}

\textsc{A letter} is the first Principle, or least part, of a Word.

An Articulate Sound is the sound of the human voice, formed by the
organs of speech.

A Vowel is a simple articulate sound, formed by the impulse of the
voice, and by the opening only of the mouth in a particular manner.

A Consonant cannot be perfectly sounded by itself; but joined with a
vowel forms a compound articulate sound, by a particular motion or
contact of the parts of the mouth.

A Diphthong, or compound vowel, is the union of two or more vowels
pronounced by a single impulse of the voice.

In English there are twenty-six Letters.

A, a; B, b; C, c; D, d; E, e; F, f; G, g; H, h; I, i; J, j; K, k; L, l;
M, m; N, n; O, o; P, p; Q, q; R, r; S, s; T, t; U, u; V, v; W, w; X, x;
Y, y; Z, z.

\tl{J j}, and \tl{V v}, are consonants; the former having the sound of
the soft \tl{g}, and the latter that of a coarser \tl{f}: they are
therefore entirely different from the vowels \tl{i} and \tl{u}, and
distinct letters of themselves; they ought also to be distinguished from
them, each by a peculiar Name; the former may be called \tl{ja}, and the
latter \tl{vee}.

The Names then of the twenty-six letters will be as follows: \tl{a, bee,
cee, dee, e, ef, gee, aitch, i, ja, ka, el, em, en, o, pee, cue, ar,
ess, tee, u, vee, double u, ex, y, zad}.

Six of the letters are vowels, and may be sounded by themselves; \tl{a,
  e, i, o, u, y}.

\tl{E} is generally silent at the end of a word; but it has its effect
in lengthening the preceding vowel, as \tl{bid}, \tl{bide}: and
sometimes likewise in the middle of a word; as, \tl{ungrateful},
\tl{retirement}. Sometimes it has no other effect, than that of
softening a preceding \tl{g}: as, \tl{lodge}, \tl{judge},
\tl{judgement}; for which purpose it is quite necessary in these and the
like words.

\tl{Y} is in sound wholly the same with \tl{i}; and is written instead
of it at the end of words; or before \tl{i}, as \tl{flying},
\tl{denying}: it is retained likewise in some words derived from the
Greek; and it is always a vowel.\footnote{The same sound, which we
  express by the initial \tl{y}, our Saxon Ancestors in many instances
  expressed by the vowel \tl{e}; as \tl{eower}, \tl{your}: and by the
  vowel \tl{i}; as \tl{iw}, \tl{yew}; \tl{iong}, \tl{young}. In the word
  \tl{yew}, the initial \tl{y} has precisely the same sound with \tl{i}
  in the words \tl{view}, \tl{lieu}, \tl{adieu}: the \tl{i} is
  acknowledged to be a Vowel in these latter; how then can the \tl{y},
  which has the very same sound, possibly be a Consonant in the former?
  Its initial sound is generally like that of \tl{i} in \tl{shire}, or
  \tl{ee} nearly: it is formed by the opening of the mouth, without any
  motion or contract of the parts: in a word, it has every property of a
  Vowel, and not one of a Consonant.}

\tl{W} is either vowel, or diphthong; its proper sound is the same as
the Italian \tl{u}, the French \tl{ou}, or the English \tl{oo}; after
\tl{o}, it is sometimes not sounded at all; sometimes like a single
\tl{u}.

The rest of the letters are consonants; which cannot be sounded alone:
some not at all, and these are called Mutes; \tl{b, c, d, g, k, p, q,
  t}: others very imperfectly, making a kind of obscure sound; and these
are called Semi-vowels, or Half-vowels, \tl{l, m, n, r, f, s}; the first
four of which are also distinguished by the name of Liquids.

The Mutes and the Semi-vowels are distinguished by their names in the
Alphabet; those of the former all beginning with a consonant, \tl{bee},
\tl{cee}, \&c.; those of the latter all beginning with a vowel, \tl{ef},
\tl{ell}, \&c.

\tl{X} is a double consonant, compounded of \tl{c}, or \tl{k}, and
\tl{s}.

\tl{Z} seems not to be a double consonant in English, as it is commonly
supposed: it has the same relation to \tl{s}, as \tl{v} has to \tl{f},
being a thicker and coarser expression of it.

\tl{H} is only an Aspiration, or Breathing: and sometimes at the
beginning of a word is not sounded at all; as, \tl{an hour}, \tl{an
  honest} man.

\tl{C} is pronounced like \tl{k}, before \tl{a, o, u}; and soft, like
\tl{s}, before \tl{e, i, y}: in like manner \tl{g} is pronounced always
hard before \tl{a, o, u}; sometimes hard and sometimes soft before
\tl{i}, and \tl{y}; and for the most part soft before \tl{e}.

The English Alphabet, like most others, is both deficient and redundant;
in some cases, the same letters expressing different sounds, and
different letters expressing the same sounds.

\section{Syllables.}

\textsc{A syllable} is a sound either simple or compounded, pronounced
by a single impulse of the voice, and constituting a word or part of a
word.

Spelling is the art of reading by naming the letters singly, and rightly
dividing words into their syllables. Or, in writing, it is the
expressing of a word by its proper letters.

In spelling, a syllable in the beginning or middle of a word ends in a
vowel, unless it be followed by \tl{x}; or by two or more consonants:
these are for the most part to be separated; and at least one of them
always belongs to the preceding syllable, when the vowel of that
syllable is pronounced short. Particles in Composition, though followed
by a vowel, generally remain undivided in spelling. A mute generally
unites with a liquid following; and a liquid, or a mute, generally
separates from a mute following: \tl{le} and \tl{re} are never separated
from a preceding mute. Examples: \tl{ma-ni-fest}, \tl{ex-e-crable},
\tl{un-e-qual}, \tl{mis-ap-ply}, \tl{dis-tin-guish},
\tl{cor-res-pon-ding}.

But the best and easiest rule, for dividing the syllables in spelling,
is to divide them as they are naturally divided in a right
pronunciation; without regard to the derivation of words, or the
possible combination of consonants at the beginning of a syllable.

\section{Words.}

\textsc{Words} are articulate sounds, used by common consent as signs of
ideas or notions.

There are in English nine Sorts of Words, or, as they are commonly
called, Parts of Speech.

\begin{enumerate}
\item The \textsc{Article}; prefixed to substantives, when they are
  common names of things, to point them out, and to show how far their
  signification extends.
\item The \textsc{Substantive}, or \textsc{Noun}; being the name of any
  thing conceived to subsist, or of which we have any notion.
\item The \textsc{Pronoun}; standing instead of the noun.
\item The \textsc{Adjective}; added to the noun to express the quality
  of it.
\item The \textsc{Verb}; or Word, by way of eminence; signifying to be,
  to do, or to suffer.
\item The \textsc{Adverb}; added to verbs, and also to adjectives and
  other adverbs, to express some circumstance belonging to them.
\item The \textsc{Preposition}; put before nouns and pronouns chiefly,
  to connect them with other words, and to show their relation to those
  words.
\item The \textsc{Conjunction}; connecting sentences together.
\item The \textsc{Interjection}; thrown in to express the affection of
  the speaker, though unnecessary with respect to the construction of
  the sentence.
\end{enumerate}

\subsection{Example.}

% TODO: Add numbers above each word.
The power of speech is a faculty peculiar to man, and was bestowed on
him by his beneficent Creator for the greatest and most excellent uses;
but alas! how often do we pervert it to the worst of purposes?

In the foregoing sentence, the Words \tl{the}, \tl{a}, are Articles;
\tl{power}, \tl{speech}, \tl{faculty}, \tl{man}, \tl{creator},
\tl{uses}, \tl{purposes}, are Substantives; \tl{him}, \tl{his}, \tl{we},
\tl{it}, are Pronouns; \tl{peculiar}, \tl{beneficent}, \tl{greatest},
\tl{excellent}, \tl{worst}, are Adjectives; \tl{is}, \tl{was},
\tl{bestowed}, \tl{do}, \tl{pervert}, are Verbs; \tl{most}, \tl{how},
\tl{often}, are Adverbs; \tl{of}, \tl{to}, \tl{on}, \tl{by}, \tl{for},
are Prepositions; \tl{and}, \tl{but}, are Conjunctions; and \tl{alas} is
an Interjection.

The Substantives, \tl{power}, \tl{speech}, \tl{faculty}, and the rest,
are General, or Common, Names of things; whereof there are many sorts
belonging to the same kind; or many individuals belonging to the same
sort: as there are many sorts of power, many sorts of speech, many sorts
of faculty, many individuals of that sort of animal called man; and so
on. These general or common names are here applied in a more or less
extensive signification; according as they are used without either, or
with the one, or with the other, of the two Articles \tl{a} and
\tl{the}. The words \tl{speech}, \tl{man}, being accompanied with no
article, are taken in their largest extent; and signify all of the kind
or sort; all sorts of speech, and all men. The word \tl{faculty}, with
the article \tl{a} before it, is used in a more confined signification,
for some one out of many of that kind, for it is here implied, that
there are other faculties peculiar to man beside speech. The words
\tl{power}, \tl{creator}, \tl{uses}, \tl{purposes}, with the article
\tl{the} before them, (for \tl{his} Creator is the same as \tl{the}
Creator \tl{of him},) are used in the most confined signification, for
the things here mentioned and ascertained: \tl{the power} is not any one
indeterminate power out of many sorts, but that particular sort of power
here specified; namely, the power of speech: \tl{the creator} is the One
great Creator of man and of all things: \tl{the uses}, and \tl{the
  purposes}, are particular uses and purposes; the former are explained
to be those in particular, that are the greatest and most excellent;
such, for instance, as the glory of God, and the common benefit of
mankind; the latter to be the worst, as lying, slandering, blaspheming,
and the like.

The Pronouns, \tl{him}, \tl{his}, \tl{we}, \tl{it}, stand instead of
some of the nouns, or substantives, going before them; as, \tl{him}
supplies the place of \tl{man}; \tl{his}, of \tl{man's}; \tl{we}, of
\tl{men}, implied in the general name \tl{man}, including all men, (of
which number is the speaker;) \tl{it}, of \tl{the power}, before
mentioned. If, instead of these pronouns, the nouns for which they stand
had been used, the sense would have been the same; but the frequent
repetition of the same words would have been disagreeable and tedious:
as, The power of speech peculiar to \tl{man}, bestowed on \tl{man}, by
\tl{man's} Creator, \&c.

The Adjectives \tl{peculiar}, \tl{beneficent}, \tl{greatest},
\tl{excellent}, \tl{worst}, are added to their several substantives, to
denote the character and quality of each.

The Verb \tl{is}, \tl{was bestowed}, \tl{do pervert}, signify severally,
being, suffering, and doing. By the first is implied, that there is such
a thing as the power of speech, and it is affirmed to be of such a kind;
namely, a faculty peculiar to man: by the second it is said to have been
acted upon, or to have had something done to it; namely, to have been
bestowed on man: by the last, we are said to act upon it, or to do
something to it; namely, to pervert it.

The Adverbs, \tl{most}, \tl{often}, are added to the adjective
\tl{excellent}, and to the verb \tl{pervert}, to show the circumstance
belonging to them; namely, that of the highest degree to the former, and
that of frequency to the latter: concerning the degree of which
frequency also a question is made, by the adverb \tl{how} added to the
adverb \tl{often}.

The Prepositions \tl{of}, \tl{to}, \tl{on}, \tl{by}, \tl{for}, placed
before the substantives and pronouns, \tl{speech}, \tl{man}, \tl{him},
\&c. connect them with other words, substantives, adjectives, and verbs,
as, \tl{power}, \tl{peculiar}, \tl{bestowed}, \&c. and show the relation
which they have to those words; as the relation of subject, object,
agent, end; \tl{for} denoting the end, \tl{by} the agent, \tl{on} the
object; \tl{to} and \tl{of} denote possession, or the belonging of one
thing to another.

The Conjunctions \tl{and}, and \tl{but}, connect the three parts of the
sentence together; the first more closely, both with regard to the
sentence and the sense; the second connecting the parts of the sentence,
though less strictly, and at the same time expressing an opposition in
the sense.

The Interjection \tl{alas!} expresses the concern and regret of the
speaker; and though thrown in with propriety, yet might have been
omitted, without injuring the construction of the sentence, or
destroying the sense.

\section{Article.}

\textsc{The Article} is a word prefixed to substantives, to point them
out, and to show how far their signification extends.

In English there are but two articles, \tl{a}, and \tl{the}: \tl{a}
becomes \tl{an} before a vowel, \tl{y} and \tl{w}\footnote{The
  pronunciation of \tl{y}, or \tl{w}, as part of a diphthong at the
  beginning of a word, requires such an effort in the conformation of
  the parts of the mouth, as does not easily admit of the article
  \tl{an} before them. In other cases the article \tl{an} in a manner
  coalesces with the vowel which it precedes: in this, the effort of
  pronunciation separates the article, and prevents the disagreeable
  consequence of a sensible hiatus.} excepted; and before a silent
\tl{h} preceding a vowel.

\tl{A} is used in a vague sense to point out one single thing of the
kind, in other respects indeterminate: \tl{the} determines what
particular thing is meaned.

A substantive, without any article to limit it, is taken in its widest
sense: thus \tl{man} means all mankind; as,

\begin{aquote}{Pope.}
  The proper study of mankind is man.
\end{aquote}

Where \tl{mankind} and \tl{man} may change places, without making any
alteration in the sense. \tl{A man} means some one or other of that
kind, indefinitely; \tl{the man} means, definitely, that particular man,
who is spoken of: the former therefore is called the Indefinite, the
latter the Definite, Article.\footnote{``And I persecuted this way unto
  \tl{the} death.'' Acts xxii. 4. The Apostle does not mean any
  particular sort of death, but death in general: the Definite Article
  therefore is improperly used. It ought to be \tl{unto death}, without
  any Article: agreeably to the Original. See also 2 Chron. xxxii. 24.

  ``When He, the Spirit of Truth, is come, he will guide you into
  \tl{all truth}.'' John xvi. 13. That is, according to this
  translation, into all Truth whatsoever, into Truth of all kinds: very
  different from the meaning of the Evangelist, and from the Original,
  into all \tl{the} Truth; that is, into all Evangelical Truth.

  ``Truly this was \tl{the} Son of God.'' Matt. xxvii. 54. and Mark xv.
  39. This translation supposes, that the Roman Centurion had a proper
  and adequate notion of the character of Jesus, as the Son of God in a
  peculiar and incommunicable sense: whereas, it is probable, both from
  the circumstances of the History, and from the expression of the
  Original, (\tl{a} Son of God, or, of \tl{a} God, not \tl{the} Son,)
  that he only meaned to acknowledge him to be an extraordinary person,
  and more than a mere man; according to his own notion of Sons of Gods
  in the Pagan Theology. This is also more agreeable to St. Luke's
  account of the same confession of the Centurion: ``Certainly this was
  a righteous man;'' not, the Just One. The same may be observed of
  Nebuchadnezzar's words, Dan. iii. 25. ``And the form of the fourth is
  like \tl{the} Son of God:'' it ought to be expressed by the Indefinite
  Article, like \tl{a} Son of God; as Theodotion very properly renders
  it: that is, like an Angel; according to Nebuchadnezzar's own account
  of it in the 28th verse: ``Blessed be God, who hath sent his
  \tl{Angel}, and delivered his servants.'' See also Luke, xix. 9.

  \begin{aquote}{Pope.}
    Who breaks a butterfly upon \tl{a} wheel?
  \end{aquote}

  It ought to be, \tl{the} wheel; used as an instrument for the
  particular purpose of torturing Criminals: as Shakspeare;

  \begin{quote}
    Let them pull all about mine ears; present me\\
    Death on \tl{the} wheel, or at wild horses heels.
  \end{quote}

  ``God Almightly hath given reason to \tl{a} man to be a light unto
  him.'' Hobbes, Elements of Law, Part I. Chap. v. 12. It should rather
  be, ``to man,'' in general.

  These remarks may serve to show the great importance of the proper use
  of the Article; the near affinity there is between the Greek Article
  and the English Definite Article; and the excellence of the English
  Language in this respect, which by means of its two Articles does most
  precisely determine the extent of signification of Common Names:
  whereas the Greek has only one Article, and it has puzzled all the
  Grammarians to reduce the use of that to any clear and certain rules.}

Example: ``\tl{Man} was made for society, and ought to extend his good
will to all \tl{men}: but \tl{a man} will naturally entertain a more
particular kindness for \tl{the men}, with whom he has the most frequent
intercourse; and enter into a still closer union with \tl{the man},
whose temper and disposition suit best with his own.''

It is of the nature of both the articles to determine or limit the thing
spoken of: \tl{a} determines it to be one single thing of the kind,
leaving it still uncertain which; \tl{the} determines which it is, or,
of many, which they are. The first therefore can only be joined to
Substantives in the singular number;\footnote{``A good character should
  not be rested in as an end, but employed as \tl{a means} of doing
  still further good.'' Atterbury, Serm. II. 3. Ought it not to be \tl{a
  man}? ``I have read an author of this taste, that compares a ragged
coin to \tl{a} tattered \tl{colors}.'' Addison, Dial. I. on Medals.};
the last may also be joined to plurals.

There is a remarkable exception to this rule in the use of the
Adjectives \tl{few} and \tl{many}, (the latter chiefly with the word
\tl{great} before it,) which, though joined with plural Substantives,
yet admit of the singular Article \tl{a}: as, \tl{a few men}, \tl{a
  great many men}:

\begin{aquote}{Shakspeare.}
  Told of \tl{a many thousand} warlike French:\\
  A care-craz'd mother of \tl{a many children}.
\end{aquote}

The reason of it is manifest from the effect, which the article has in
these phrases: it means a small or great number collectively taken, and
therefore gives the idea of a Whole, that is, of Unity.\footnote{Thus
  the word \tl{many} is taken collectively as a Substantive:

  \begin{aquote}{Shakspeare, 2 Henry IV.}
    O Thou fond \tl{Many!} with what loud applause\\
    Didst thou beat heav'n with blessing Bolingbroke,\\
    Before he was what thou wouldst have him be!
  \end{aquote}

  But it will be hard to reconcile to any Grammatical propriety the
  following phrase: ``\tl{Many one} there \tl{be}, that say of my soul;
  There is no help for him in his God.'' Psal. iii. 2.

  \begin{aquote}{Swift, Verses on his own Death.}
    \tl{How many a message} would he send!
  \end{aquote}

  ``He would send \tl{many a message},'' is right: but the question
  \tl{how} seems to destroy the unity, or collective nature, of the
  Idea; and therefore it ought to have been expressed, if the measure
  would have allowed of it, without the article, in the plural number;
  \tl{how many messages}.} Thus likewise \tl{a hundred}, \tl{a
  thousand}, is one whole number, an aggregate of many collectively
taken; and therefore still retains the article \tl{a}, though joined as
an Adjective to a plural Substantive; as, \tl{a hundred
  years}.\footnote{``There were slain of them upon \tl{a} three thousand
men:'' that is, to the number of three thousand. I Macc. iv. 15. ``About
\tl{an} eight Days:'' that is, a space of eight days. Luke, ix. 28. But
the expression is obsolete, or at least vulgar; and, we may add
likewise, improper: for neither of these numbers has been reduced by use
and convenience into one collective and compact idea, like \tl{a
  hundred}, and \tl{a thousand}; each of which, like \tl{a dozen}, or
\tl{a score}, we are accustomed equally to consider on certain occasions
as a simple Unity.}

\begin{aquote}{Dryden.}
  For harbour at \tl{a thousand doors} they knock'd;\\
  Not one of all \tl{the thousand}, but was lock'd.
\end{aquote}

The Definite Article \tl{the} is sometimes applied to Adverbs in the
Comparative and Superlative degree; and its effect is to mark the degree
the more strongly, and to define it the more precisely: as, ``\tl{The
  more} I examine it, \tl{the better} I like it. I like this \tl{the
  least} of any.''

\section{Substantive.}

\textsc{A Substantive}, or \textsc{Noun}, is the \tl{Name} of a thing;
of whatever we conceive in any way to \tl{subsist}, or of which we have
any notion.

Substantives are of two sorts; Proper, and Common, Names. Proper Names
are the Names appropriated to individuals; as the names of persons and
places: such are \tl{George}, \tl{London}. Common Names stand for kinds,
containing many sorts; or for sorts, containing many individuals under
them; as, \tl{Animal}, \tl{Man}. And these Common Names, whether of
kinds or sorts, are applied to express individuals, by the help of
Articles added to them, as hath been already shown; and by the help of
Definite Pronouns, as we shall see hereafter.

Proper Names being the Names of individuals, and therefore of things
already as determinate as they can be made, admit not of Articles, or of
Plurality of number; unless by a Figure, or by Accident; as, when great
Conquerors are called \tl{Alexanders}; and some great Conqueror \tl{An}
Alexander, or \tl{The} Alexander of his Age: when a Common Name is
understood, as \tl{The} Thames, that is, the \tl{River} Thames; \tl{The}
George, that is, the \tl{Sign} of St. George: or when it happens, that
there are many persons of the same name; as, \tl{The} two \tl{Scipios}.

Whatever is spoken of is represented as one, or more, in Number: these
two manners of representation in respect of number are called the
Singular, and the Plural, Number.

In English, the Substantive Singular is made Plural, for the most part,
by adding to it \tl{s}; or \tl{es}, where it is necessary for the
pronunciation: as \tl{king}, \tl{kings}; \tl{fox}, \tl{foxes};
\tl{leaf}, \tl{leaves}; in which last, and many others, \tl{f} is also
changed into \tl{v}, for the sake of an easier pronunciation, and more
agreeable sound.

Some few Plurals end in \tl{en}; as, \tl{oxen}, \tl{children},
\tl{brethren}, and \tl{men}, \tl{women}, by changing the \tl{a} of the
Singular into \tl{e}.\footnote{And anciently, \tl{eyen}, \tl{shoen},
  \tl{housen}, \tl{hosen}: so likewise anciently \tl{sowen}, \tl{cowen},
  now always pronounced and written \tl{swine}, \tl{kine}.} This form we
have retained from the Teutonic; as likewise the introduction of the
\tl{e} in the former syllable of two of the last instances; \tl{weomen},
(for so we pronounce it,) \tl{brethren}, form \tl{woman},
\tl{brother}:\footnote{In the German, the vowels, \tl{a}, \tl{o},
  \tl{u}, of monosyllable Nouns, are generally in the Plural changed
  into diphthongs with an \tl{e}: as die \tl{hand}, the hand, die
  \tl{h\"{a}nde}; der \tl{hut}, the hat, die \tl{h\"{u}te}; der
  \tl{knopff}, the button, (or knob,) die \tl{kn\"{o}pffe}; \&c.}
sometimes like which may be noted in some other forms of the Plurals: as
\tl{mouse}, \tl{mice}; \tl{louse}, \tl{lice}; \tl{tooth}, \tl{teeth};
\tl{foot}, \tl{feet}; \tl{goose}, \tl{geese}.\footnote{These are
  directly from the Saxon: \tl{mus}, \tl{mys}; \tl{lus}, \tl{lys};
  \tl{toth}, \tl{teth}; \tl{fot}, \tl{fet}; \tl{gos}, \tl{ges}.}

The words \tl{sheep}, \tl{deer}, are the same in both Numbers.

Some Nouns, from the nature of the things which they express, are used
only in the Singular, others only in the Plural, Form: as \tl{wheat},
\tl{pitch}, \tl{gold}, \tl{sloth}, \tl{pride}, \&c. and \tl{bellows},
\tl{scissars}, \tl{lungs}, \tl{bowels}, \&c.

The English Language, to express different connexions and relations of
one thing to another, uses, for the most part, Prepositions. The Greek
and Latin among the ancient, and some too among the modern languages, as
the German, vary the termination or ending of the Substantive, to answer
the same purpose. These different endings are in those languages called
Cases. And the English being derived from the same origin as the German,
that is, from the Teutonic,\footnote{``Lingua Anglorum hodierna
  avit\ae{} Saxonic\ae{} formam in plerisque orationis partibus etiamnum
  retinet. Nam quoad particulas casuales, quorundam casuum
  terminationes, conjugationes verborum, verbum substantivum, formam
  passiv\ae{} vocis, pronomina, participia, conjunctiones, \&
  pr\ae{}positiones omnes; denique, quoad idiomata, phrasiumque maximam
  partem, etiam nunc Saxonicus est Anglorum sermo.'' Hickes, Thesaur.
  Ling. Septent. Pr\ae{}f. p. vi. To which may be added the Degrees of
  Comparison, the form of which is the very same in the English as in
  the Saxon.} is not wholly without them. For instance, the relation of
Possession, or Belonging, is often expressed by a Case, or a different
ending of the Substantive. This Case answers to the Genitive Case in
Latin, and may still be so called, though perhaps more properly the
Possessive Case. Thus, ``\tl{God's} grace:'' which may also be expressed
by the Preposition; as, ``the grace \tl{of God}.'' It was formerly
written, ``\tl{Godis} grace;'' we now always shorten it with an
Apostrophe; often very improperly, when we are obliged to pronounce it
fully; as, ``\tl{Thomas's} book:'' that is, ``\tl{Thomasis} book,'' not
``\tl{Thomas his} book,'' as it is commonly
supposed.\footnote{``\tl{Christ his} sake,'' in our Liturgy, is a
  mistake, either of the Printers, or of the Compilers. ``Nevertheless,
  Asa \tl{his} heart was perfect with the Lord.'' I Kings, xv. 14. ``To
  see whether Mordecai \tl{his} matters would stand.'' Esther, iii. 4.

  \begin{aquote}{Donne.}
    Where is this mankind now? who lives to age\\
    Fit to be made Methusalem \tl{his} page?
  \end{aquote}

  \begin{aquote}{Pope's Odyssey.}
    By young Telemachus \tl{his} blooming years.
  \end{aquote}

  ``My Paper is the \tl{Ulysses his} bow, in which every man of wit or
  learning may try his strength.'' Addison, Guardian, No. 98. See also
  Spect. No. 207. This is no slip of Mr. Addison's pen: he gives us his
  opinion upon this point very explicitly in another place. ``The same
  single letter \tl{s} on many occasions does the office of a whole
  word; and represents the \tl{his} or \tl{her} of our forefathers.''
  Addison, Spect. No. 135. The latter instance might have shown him, how
  groundless this notion is: for it is not easy to conceive, how the
  letter \tl{s} added to a Feminine Noun should represent the word
  \tl{her}; any more than it should the word \tl{their}, added to a
  Plural Noun; as, ``the \tl{children's} bread.'' But the direct
  derivation of this Case from the Saxon Genitive Case is sufficient of
  itself to decide this matter.}

% TODO: Footnotemark is missing a footnote here in the original
When the thing, to which another is said to belong, is expressed by a
circumlocution, or by many terms, the sign of the Possessive Case is
commonly added to the last term; as, ``The King of Great \tl{Britain's}
Soldiers.'' When it is a Noun ending in \tl{s}, the sign of the
Possessive Case is sometimes not added; as, ``for \tl{righteousness'
  sake};\footnote{In Poetry, the Sign of the Possessive Case is
  frequently omitted after Proper Names ending in \tl{s}, or \tl{x}: as,
``The wrath of Peleus' Son.'' Pope. This seems not so allowable in
Prose: as, ``Moses' minister.'' Josh. i. 1. ``Phinehas' wife.'' I Sam.
iv. 19. ``Festus came into Felix' room.'' Acts, xxiv. 27.}'' nor ever to
the Plural Number ending in \tl{s}; as, ``on \tl{eagles'}
wings.\footnotemark'' Both the Sign and the Preposition seem sometimes
to be used; as, ``a soldier \tl{of the king's}:'' but here are really
two Possessives; for it means, ``one \tl{of} the soldiers \tl{of} the
king.''

The English in its Substantives has but two different terminations for
Cases; that of the Nominative, which simply expresses the Name of the
thing, and that of the Possessive Case.

Things are frequently considered with relation to the distinction of Sex
or Gender; as being male, or Female, or Neither the one, nor the other.
Hence Substantives are of the Masculine, or Feminine, or Neuter, (that
is, Neither,) Gender: which latter is only the exclusion of all
consideration of Gender.

The English Language, with singular propriety, following nature alone,
applies the distinction of Masculine and Feminine only to the names of
Animals; all the rest are Neuter: except whem, by a Poetical or
Rhetorical fiction, things Inanimate and Qualities are exhibited as
Persons, and consequently become either Male or Female. And this gives
the English an advantage above most other languages in the Poetical and
Rhetorical style: for when Nouns naturally Neuter are converted into
Masculine and Feminine,\footnote{
  \begin{aquote}{Milton, P. L. B. vi.}
    At his command th' uprooted Hills retired\\
    Each to \tl{his} place: they heard his voice, and went\\
    Obsequious: Heaven \tl{his} wonted face renew'd,\\
    And with fresh flowrets Hill and Valley smil'd.
  \end{aquote}

  \begin{aquote}{Milton, Comus.}
    Was I deceiv'd; or did a sable Cloud\\
    Turn forth \tl{her} silver lining on the Night?
  \end{aquote}

  ``Of Law no less can be acknowledged, than that \tl{her} seat is the
  bosom of God; \tl{her} voice, the harmony of the world. All things in
  heaven and earth do \tl{her} homage: the very least, as feeling
  \tl{her} care; and the greatest, as not exempted from \tl{her}
  power.'' Hooker, B. i. 16. ``Go to your Natural Religion: lay before
  \tl{her} Mahomet and his disciples, arrayed in armour and in
  blood:---show \tl{her} the cities, which he set in flames; the
  countries, which he ravaged:---when \tl{she} has viewed them in this
  scene, carry \tl{her} into his retirements; show \tl{her} the
  Prophet's chamber, his concubines and his wives:---when \tl{she} is
  tired with this prospect, then show \tl{her} the Blessed Jesus.'' See
  the whole passage in the conclusion of Bp. Sherlock's 9th Sermon, vol.
  i.

  Of these beautiful passages we may observe, that as, in the English,
  if you put \tl{it} and \tl{its} instead of \tl{his}, \tl{she},
  \tl{her}, you confound and destroy the images, and reduce, what was
  before highly Poetical and Rhetorical, to mere prose and common
  discourse; so if you render them into another language, Greek, Latin,
  French, Italian, or German; in which Hill, Heaven, Cloud, Law,
  Religion, are constantly Masculine, or Feminine, or Neuter,
  respectively; you make the images obscure and doubtful, and in
  proportion diminish their beauty.

  This excellent remark is Mr. Harris's, \textsc{Hermes}, p. 38.}, this
Personification is more distinctly and forcibly marked.

Some few Substantives are distinguished in their Gender by their
terminations: as, \tl{prince}, \tl{princess}; \tl{actor}, \tl{actress};
\tl{lion}, \tl{lioness}; \tl{hero}, \tl{heroine}; \&c.

The chief use of Gender in English is in the Pronoun of the Third
Person; which must agree in that respect with the Noun for which it
stands.

\section{Pronoun.}

\textsc{A Pronoun} is a word standing \tl{instead of a Noun}, as its
Substitute or Representative.

In the Pronoun are to be considered the Person, Number, Gender, and
Case.

There are Three Persons which may be the Subject of any discourse:
first, the Person who speaks may speak of himself; secondly, he may
speak of the Person to whom he addresses himself; thirdly, he may speak
of some other Person.

These are called, respectively, the First, Second, and Third, Persons:
and are expressed by the Pronouns, \tl{I}, \tl{Thou}, \tl{He}.

As the Speakers, the Persons spoken to, and the other Persons spoken of,
may be many; so each of these Persons hath the Plural Number; \tl{We},
\tl{Ye}, \tl{They}.

The Persons speaking and spoken to, being at the same time the Subjects
of discourse, are supposed to be present; from which and other
circumstances their Sex is commonly known, and need not to be marked by
a distinction of Gender in their Pronouns: but the third Person or thing
spoken of being absent and in many respects unknown, it is necessary,
that it should be marked by a distinction of Gender; at least when some
particular person or thing is spoken of, which ought to be more
distinctly marked: accordingly the Pronoun Singular of the Third Person
hath the Three Genders; \tl{He}, \tl{She}, \tl{It}.

Pronons have Three Cases; the Nominative; the Genitive, or Possessive;
like Nouns; and moreover a Case, which follows the Verb Active, or the
Preposition, expressing the Object of an Action, or of a Relation. It
answers to the Oblique Cases in Latin; and may be properly enough called
the Oblique Case.

\begin{center}
  \textsc{pronouns};

  according to their Persons, Numbers, Cases,\\
  and Genders.

  \textsc{persons}.

  \begin{tabular}[h]{cccccc}
    1. & 2. & 3. & 1. & 2. & 3.\\
    \multicolumn{3}{c}{Singular.} & \multicolumn{3}{c}{Plural.}\\
    I, & Thou, & He; & We, & Ye or You, & They.\\
  \end{tabular}

  \textsc{cases}.

  \begin{tabular}[h]{cccccc}
    Nom. & Poss. & Obj. & Nom. & Poss. & Obj.\\
    \multicolumn{6}{c}{First Person.}\\
    I, & Mine, & Me; & We, & Ours, & Us.\\
    \multicolumn{6}{c}{Second Person.}\\
    Thou, & Thine, & Thee; & Ye or You, & Yours, & You.\footnotemark\\
  \end{tabular}

  % TODO: This does not fit with the following third-person tabbing
  Third Person.
\end{center}
\footnotetext{Some Writers have used \tl{Ye} as the Objective Case
  Plural of the Pronoun of the Second Person; very improperly, and
  ungrammatically.

  \begin{aquote}{Shakspeare, Hen. VIII.}
    The more shame for \tl{ye}: holy men I thought \tl{ye}.
  \end{aquote}

  \begin{aquote}{Prior.}
    But tyrants dread \tl{ye}, lest your just degree\\
    Transfer the pow'r, and set the people free.
  \end{aquote}

  \begin{aquote}{Milton, P. L. ii. 734.}
    His wrath, which one day will destroy \tl{ye} both.
  \end{aquote}

  Milton uses the same manner of expression in a few other places of his
  Paradise Lost, and more frequently in his Poems. It may perhaps be
  allowed in the Comic and Burlesque style, which often imitates a
  vulgar and incorrect pronunciation: as, ``By the Lord, I knew \tl{ye},
  as well as he that made \tl{ye}.'' Shakspeare, I Henry IV. But in the
  serious and solemn style, no authority is sufficient to justify so
  manifest a solecism.

  The Singular and Plural Forms seem to be confounded in the following
  Sentence: ``Pass \tl{ye} away, \tl{thou} inhabitant of Saphir.''
  Micah, i. II.}

\begin{tabbing}
  \tl{Mas.}\hspace*{0.5cm} \={}He, His, Him;\\
  \tl{Fem.} \>{}She, Hers, Her;\hspace*{0.5cm} They, Theirs, Them.\\
  \tl{Neut.} \>{}It, Its\footnotemark, It;\\
\end{tabbing}
\footnotetext{The Neuter Pronoun of the Third Person had formerly no
  variation of Cases. Instead of the Possessive \tl{its} they used
  \tl{his}, which is now appropriated to the Masculine, ``Learning hath
  \tl{his} infancy, when \tl{it} is but beginning, and almost childish;
  then \tl{his} youth, when \tl{it} is luxuriant and juvenile; then
  \tl{his} strength of years, when \tl{it} is solid and reduced; and
  lastly \tl{his} old age, when \tl{it} waxeth dry and exhaust.'' Bacon,
  Essay 58. In this example \tl{his} is evidently used as the Possessive
  Case of \tl{it}: but what shall we say to the following, where
  \tl{her} is applied in the same manner, and seems to make a strange
  confusion of Gender? ``He that pricketh the heart maketh \tl{it} to
  show \tl{her} knowledge.'' Eccles, xxii. 19.

  \begin{aquote}{Shakspeare, 2 Hen. VI.}
    Oft have I seen a timely-parted ghost,\\
    Of ashy semblence, meagre, pale, and bloodless,\\
    Being all descended to the lab'ring heart,\\
    \tl{Who}, in the conflict that \tl{it} holds with death,\\
    Attracts the same for aidance 'gainst the enemy.
  \end{aquote}

  It ought to be,

  \begin{quote}
    \tl{Which}, in the conflict that \tl{it} holds---
  \end{quote}

  Or, perhaps more poetically,

  \begin{quote}
    \tl{Who}, in the conflict that \tl{he} golds with death.
  \end{quote}}

The Personal Pronouns have the nature of Substantives, and, as such,
stand by themselves: the rest have the nature of Adjectives, and, as
such, are joined to Substantives; and may be called Pronominal
Adjectives.

\tl{Thy}, \tl{My}, \tl{Her}, \tl{Our}, \tl{Your}, \tl{Their}, are
Pronominal Adjectives: but \tl{His}, (that is, \tl{He's}) \tl{Her's},
\tl{Our's}, \tl{Your's}, \tl{Their's}, have evidently the Form of the
Possessive Case: and by Analogy, \tl{Mine}, \tl{Thine},\footnote{So the
  Saxon \tl{Ic} hath the Possessive Case \tl{Min}; \tl{Thu}, Possessive
  \tl{Thin}; \tl{He}, Possessive \tl{His}: from which our Possessive
  Cases of the same Pronouns are taken without Alteration. To the Saxon
  Possessive Cases, \tl{hire}, \tl{ure}, \tl{eower}, \tl{hira}, (that
  is, \tl{her's}, \tl{our's}, \tl{your's}, \tl{their's},) we have added
  the \tl{s}, the Characteristic of the Possessive Case of Nouns. Or
  \tl{our's}, \tl{your's}, are directly from the Saxon \tl{ures},
  \tl{eowers}; the Possessive Case of the Pronominal Adjectives
  \tl{ure}, \tl{eower}; that is, \tl{our}, \tl{your}.} may be esteemed
of the same rank. All these are used, when the Noun, to which they
belong, is understood: the two latter sometimes also instead of \tl{my},
\tl{thy}, when the Noun following them begins with a vowel.

Beside the foregoing, there are several other Pronominal Adjectives;
which, though they may sometimes seem to stand by themselves, yet have
always some Substantive belonging to them, either referred to, or
understood: as, \tl{This}, \tl{that}, \tl{other}, \tl{any}, \tl{same},
\tl{one}, \tl{none}. These are called Definitive, because they
\tl{define} and limit the extent of the Common Name, or General Term, to
which they either refer, or are joined. The three first of these are
varied, to express Number; as, \tl{These}, \tl{those},
\tl{others};\footnote{``Diodorus, whose design was to refer to all
  occurrences to years,---is of more credit in a point of Chronology,
  than Plutarch or any \tl{other}, that \tl{write} Lives by the lump.''
  Bentley, Dissert. on Themistocles's Epistles, Sect. vi. It ought to be
\tl{others}, or \tl{writes}.} the last of which admits of the Plural
form only when its Substantive is not joined to it, but referred to, or
understood: none of them are varied to express the Gender; only two of
them to express the Case; as, \tl{other}, \tl{one}, which have the
Possessive Case. \tl{One} is sometimes used in an Indefinite sense,
(answering to the French \tl{on},) as in the following phrases;
``\tl{one} is apt to think;'' ``\tl{one} sees;'' ``\tl{one} supposes.''
\tl{Who}, \tl{which}, \tl{that}, are called Relatives, because they more
directly \tl{refer} to some substantive \tl{going before}; which
therefore is called the Antecedent. They also connect the following part
of the Sentence with the foregoing. These belong to all the three
Persons; whereas the rest belong only to the Third. One of them only is
varied to express the three Cases; \tl{Who},
\tl{whose},\footnote{\tl{Whose} is by some authors made the Possessive
  Case of \tl{which}, and applied to things as well as persons: I think
  improperly.

  \begin{aquote}{Dryden.}
    The \tl{question}, \tl{whose} solution I require,\\
    Is, what the sex of women most desire.\\
  \end{aquote}

  \begin{aquote}{Addison.}
    Is there any other \tl{doctine}, \tl{whose} followers are punished?
  \end{aquote}

  The higher Poetry, which lives to consider every thing as bearing a
  Personal Character, frequently applies the personal Possessive
  \tl{whose} to inanimate beings:

  \begin{aquote}{Milton.}
    Of man's first disobedience, and the fruit\\
    Of that forbidden Tree, \tl{whose} mortal taste\\
    Brought death into the world, and all our woe.
  \end{aquote}}(that is, \tl{who's},\footnote{So the Saxon \tl{hwa} hath
  the Possessive Case, \tl{hw\ae{}s}. Note, that the Saxons rightly
  placed the Aspirate before the \tl{w}: as we now pronounce it. This
  will be evident to any one that shall consider in what manner he
  pronounces the words \tl{what}, \tl{when}; that is, \tl{hoo-at},
  \tl{hoo-en}.}) \tl{whom}: none of them have different endings for the
Numbers. \tl{Who}, \tl{which}, \tl{what}, are called Interrogatives,
when they are used in \tl{asking questions}. The two latter of them have
no variation of Number or Case. \tl{Each},
\tl{every},\footnote{\tl{Every} was formerly much used as a Pronominal
  Adjective, standing by itself: as, ``He proposeth unto God their
  necessities, and they their own requests, for relief in \tl{every} of
  them.'' Hooker, v. 39. ``The corruptions and depravations to which
  \tl{every} of these was subject.'' Swift, Contests and Dissentions. We
  now commonly say, \tl{every one}.} \tl{either}, are called
Distributives; because they denote the persons, or things, that make up
a number, as taken \tl{separately} and singly.

\tl{Own}, and \tl{self}, in the Plural \tl{selves}, are joined to the
Possessives, \tl{my}, \tl{our}, \tl{thy}, \tl{your},
\tl{his},\footnote{The Possessives \tl{his}, \tl{mine}, \tl{thine}, may
  be accounted either Pronominal Adjectives, or Genitive Cases of the
  respective Pronouns. The form is ambiguous; just in the same manner
  as, in the Latin phrase ``\tl{cujus} liber,'' the word \tl{cujus} may
  be either the Genitive Case of \tl{qui}, or the Nominative Masculine
  of the Adjective, \tl{cujus}, \tl{cuja}, \tl{cujum}. So likewise,
  \tl{mei}, \tl{tui}, \tl{sui}, \tl{nostri}, \tl{vestri}, have the same
  form, whether Pronouns, or Pronominal Adjectives.} \tl{her},
\tl{their}; as, \tl{my own} hand; \tl{myself}, \tl{yourselves}: both of
them expressing emphasis, or opposition; as, ``I did it \tl{my own
  self},'' that is, and no one else: that latter also forming the
Reciprocal Pronoun; as, ``he hurt \tl{himself}.'' \tl{Himself},
\tl{themselves}, seem to be used in the Nominative Case by corruption
instead of \tl{his self}, \tl{their selves}:\footnote{\tl{His self} and
  \tl{their selves} were formerly in use, even in the objective Case
  after a Preposition: ``Every of us, each for \tl{his self}, laboured
  how to recover him.'' Sidney. ``That they would willingly and of
  \tl{their selves} endeavour to keep a perpetual chastity.'' Stat. 2
  and 3 Ed. VI. ch. 21.} as, ``he came \tl{himself};'' ``they did it
\tl{themselves};'' where \tl{himself}, \tl{themselves}, cannot be in the
Objective Case. If this be so, \tl{self} must be, in these instances,
not a Pronoun, but a Noun. Thus Dryden uses it:

\begin{quote}
  What I show,\\
  Thy \tl{self may} freely on thyself bestow.
\end{quote}

\tl{Ourself}, the Plural Pronominal Adjective with the Singular
Substantive, is peculiar to the Regal Style.

\tl{Own} is an Adjective; or perhaps the Participle
\tl{owen},\footnote{Chaucer has thus expressed it:

  \begin{quote}
    As friendly, as he were his \tl{owen} brother.
  \end{quote}

  Cant. Tales, 1654, edit. 1775. And so in many other places; and, I
  believe, always in the same manner.} of the verb \tl{to owe}; to be
the right owner of a thing.\footnote{``The Man that \tl{oweth} this
  girdle.'' Acts, xxi. 11.}

All Nouns whatever in Grammatical Construction are of the Third Person;
except when an address is made to a Person: then the Noun, (answering to
what is called the Vocative Case in Latin,) is of the Second Person.

\section{Adjective.}

\textsc{An Adjective} is a word \tl{added to} a Substantive to express
its quality.\footnote{Adjectives are very improperly called \tl{Nouns};
  for they are not the \tl{Names} of things. The Adjectives \tl{good},
  \tl{white}, are applied to the Nouns \tl{man}, \tl{snow}, to express
  the Qualities belonging to those Subjects; but the Names of those
  Qualities in the Abstract, (that is, considered in themselves, and
  without being attributed to any Subject,) are \tl{goodness},
  \tl{whiteness}; and these are Nouns, or Substantives.}

In English the Adjective is not varied on account of Gender, Number, or
Case.\footnote{Some few Pronominal Adjectives must here be excepted, as
  having the Possessive Case; as \tl{one}, \tl{other}, \tl{another}:

  \begin{aquote}{Sidney.}
    By \tl{one's} own choice.
  \end{aquote}

  \begin{aquote}{Pope, Univ. Prayer.}
    Teach me to feel \tl{another's} woe.
  \end{aquote}

  And the Adjectives, \tl{former}, and \tl{latter}, may be considered as
  Pronominal, and representing the Nouns, to which they refer; if the
  phrase in the following sentence be allowed to be just: ``It was happy
  for the state, that Fabius continued in the command with Minucius: the
  \tl{former's} phlegm was a check upon the \tl{latter's} vivacity.} The
only variation, which it admits of, is that of the Degrees of
Comparison.

Qualities for the most part admit of \tl{more} and \tl{less}, or of
different degrees: and the words that express such Qualities have
accordingly proper forms to express different degrees. When a Quality is
simply expressed without any relation to the same in a different degree,
it is called the Positive; as, \tl{wise}, \tl{great}. When it is
expressed with augmentation, or with reference to a less degree of the
same, it is called the Comparative; as, \tl{wiser}, \tl{greater}. When
it is expressed as being in the highest degree of all, it is called the
Superlative; as, \tl{wisest}, \tl{greatest}.

So that the simple word, or Positive, becomes Comparative by adding
\tl{r} or \tl{er}; and Superlative by adding \tl{st} or \tl{est}, to the
end of it. And the Adverbs \tl{more} and \tl{most} placed before the
Adjective have the same effect; as, \tl{wise}, \tl{more wise}, \tl{most
  wise}.\footnote{Double Comparatives and Superlatives are improper:

  \begin{aquote}{Shakspeare, Tempest.}
    The Duke of Milan,\\
    And his \tl{more braver} Daughter could controul thee.
  \end{aquote}

  ``After the \tl{most straitest} sect of our religion I have lived a
  Pharisee.'' Acts, xxvi. 5. So likewise Adjectives, that have in
  themselves a Superlative signification, admit not properly the
  Superlative form superadded: ``Whosoever of you will be \tl{chiefest},
  shall be servant of all:'' Mark, x. 44. ``One of the first and
  \tl{chiefest} instances of prudence.'' Atterbury, Serm. IV. 10.
  ``While the \tl{extremest} parts of the earth were meditating a
  submission.'' Ibid. I. 4.

  \begin{aquote}{Milton, Il Penseroso.}
    But first and \tl{chiefest} with thee bring\\
    Him, that yon soars on golden wing,\\
    Guiding the fiery-wheeled throne,\\
    The Cherub Contemplation.
  \end{aquote}

  \begin{aquote}{Addison's Travels.}
    That on the sea's \tl{extremest} border stood.
  \end{aquote}

  But poetry is in possession of these two improper Superlatives, and
  may be indulged in the use of them.

  The Double Superlative \tl{most highest} is a Phrase peculiar to the
  Old Vulgar Translation of the Psalms; where it acquires a singular
  propriety from the Subject to which it is applied, the Supreme Being,
  who is \tl{higher than the highest}.}

Monosyllables, for the most part, are compared by \tl{er} and \tl{est};
and Disyllables by \tl{more} and \tl{most}; as, \tl{mild}, \tl{milder},
\tl{mildest}; \tl{frugal}, \tl{more frugal}, \tl{most frugal}.
Disyllables ending in \tl{y}, \tl{happy}, \tl{lovely}; and in \tl{le}
after a mute, as \tl{able}, \tl{ample}; or accented on the last
syllable, as \tl{discrete}, \tl{polite}; easily admit of \tl{er} and
\tl{est}. Words of more than two syllables hardly ever admit of those
terminations.

In some few words the Superlative is formed by adding the Adverb
\tl{most} to the end of them: as, \tl{nethermost}, \tl{uttermost}, or
\tl{utmost}, \tl{undermost}, \tl{uppermost}, \tl{foremost}.

In English, as in most languages, there are some words of very common
use, (in which the caprice of Custom is apt to get the better of
Analogy,) that are irregular in this respect: as \tl{good}, \tl{better},
\tl{best}; \tl{bad}, \tl{worse}, \tl{worst}; \tl{little},
\tl{less},\footnote{``\tl{Lesser}, says Dr. Johnson, is a barbarous
  corruption of \tl{less}, formed by the vulgar from the habit of
  terminating Comparisons in \tl{er}.''

  % TODO: Does the above go with Addison?
  \begin{aquote}{Addison.}
    Attend to what a \tl{lesser} Muse indites.
  \end{aquote}

  ``The tongue is like a race-horse; which runs the faster, the
  \tl{lesser} weight is carries.'' Addison, Spect. No. 247.

  \tl{Worser} sounds much more barbarous, only because it has not been
  so frequently used.

  \begin{aquote}{Shakspeare, 1 Hen. VI.}
    Changed to a \tl{worser} shape thou canst not be.
  \end{aquote}

  \begin{aquote}{Dryden.}
    A dreadful quiet felt, and \tl{worser} far\\
    Than arms, a sullen interval of war.
  \end{aquote}

  The Superlative \tl{least} ought rather to be written without the
  \tl{a}, being contracted from \tl{lessest}; as Dr. Wallis hath long
  ago observed. The Conjunction, of the same sound, might be written
  with the \tl{a}, for distinction.} \tl{least}; \tl{much}, or
\tl{many}, \tl{more}, \tl{most}; and a few others. And in other
languages, the words irregular in this respect are those which express
the very same ideas with the foregoing.

\section{Verb.}

\textsc{A Verb} is a \tl{word} which signifies to be, to do, or to
suffer.

There are three kinds of Verbs; Active, Passive, and Neuter Verbs.

A Verb Active expresses an Action, and necessarily implies an Agent, and
an Object acted upon: as, \tl{to love}; ``I love Thomas.''

A Verb Passive expresses a Passion, or a Suffering, or the Receiving of
an Action; and necessarily implies an Object acted upon, and an Agent by
which it is acted upon; as, \tl{to be loved}; ``Thomas is loved by me.''

So when the Agent takes the lead in the Sentence, the Verb is Active,
and is followed by the Object: when the Object takes the lead, the Verb
is Passive, and is followed by the Agent.

A Verb Neuter expresses Being; or a state or condition of being; when
the Agent and the Object acted upon coincide, and the event is properly
Neither action nor passion, but rather something between both: as, \tl{I
  am}, \tl{I sleep}, \tl{I walk}.

The Verb Active is called also Transitive; because the action
\tl{passeth over} to the Object, or hath an effect upon some other
thing: and the Verb Neuter is called Intransitive; because the effect is
confined within the Agent, and doth \tl{not pass over} to any
object.\footnote{The distinction between Verbs absolutely Neuter, as
  \tl{to sleep}, and Verbs Active Intransitive, as \tl{to walk}, though
  founded in nature and truth, is of little use in Grammar. Indeed it
  would rather perplex than assist the learner: for the difference
  between Verbs Active and Neuter, as Transitive and Intransitive, is
  easy and obvious: but the difference between Verbs absolutely Neuter
  and Intransitively Active is not always clear. But however these
  latter may differ in nature, the Construction of them both is the
  same: and Grammar is not so much concerned with their real, as with
  their Grammatical, properties.}

In English many Verbs are used both in an Active and Neuter
signification, the construction only determining of which \tl{kind} they
are.

To the signification of the Verb is superadded the designation of
Person, by which it corresponds with the several Personal Pronouns; of
Number, by which it corresponds with the Number of the Noun, Singular or
Plural; of Time, by which it represents the being, action, or passion,
as Present, Past, or Future; whether Imperfectly, or Perfectly; that is,
whether passing in such time, or then finished; and lastly of Mode, or
of the various Manner in which the being, action, or passion, is
expressed.

In a Verb therefore are to be considered the Person, the Number, the
Time, and the Mode.

The Verb in some parts of it varies its endings, to express, or agree
with, different Persons of the same number: as, ``I \tl{love}, Thou
\tl{lovest}, He \tl{loveth}, or \tl{loves}.''

So also to express different Numbers of the same person: as, ``Thou
\tl{lovest}, Ye \tl{love}; He \tl{loveth}, They \tl{love}.\footnote{In
  the Plural Number of the Verb, there is no variation of ending to
  express the different Persons; and the three Persons Plural are the
  same also with the first Person Singular: moreover in the Present Time
  of the Subjunctive Mode all Personal Variation is wholly dropped. Yet
  is this scanty provision of terminations sufficient for all the
  purposes of discourse, nor does any ambiguity arise from it: the Verb
  being always attended either with the Noun expressing the Subject
  acting or acted upon, or the Pronoun representing it. For which reason
  the Plural Termination in \tl{en}, \tl{they loven}, \tl{they weren},
  formerly in use, was laid aside as unnecessary, and hath long been
  obsolete.}

So likewise to express different Times, in which any thing is
represented as being, acting, or acted upon: as, ``I \tl{love}, I
\tl{loved}; I \tl{bear}, I \tl{bore}, I have \tl{borne}.''

The Mode is the \tl{Manner} of representing the Being, Action, or
Passion. When it is simply \tl{declared}, or a question is asked, in
order to obtain a \tl{declaration} concerning it, it is called the
Indicative Mode; as, ``I \tl{love}; \tl{lovest} thou?'' when it is
\tl{bidden}, it is called the Imperative; as, ``\tl{love} thou:'' when
it is \tl{subjoined} as the end or design, or mentioned under a
condition, a supposition, or the like, for the most part depending on
some other Verb, and having a Conjunction before it, it is called the
Subjunctive; as, ``If I \tl{love}; if thou \tl{love}:'' when it is
barely expressed \tl{without any limitation} of person or number, it is
called the Infinitive; as, ``\tl{to love};'' and when it is expressed in
a form

% TODO: Page missing
PAGE MISSING!

% TODO: Is this all a quote?
``called therefore Auxiliaries, or Helpers; \tl{do}, \tl{be}, \tl{have},
\tl{shall}, \tl{will}: as, I \tl{do} love, I \tl{did} love, I \tl{am}
loved, I \tl{was} loved; I \tl{have} loved, I \tl{have been} loved; I
\tl{shall}, or \tl{will}, loved, or \tl{be loved}.''

The two principal Auxiliaries, to \tl{have} and \tl{to be}, are thus
varied, according to Person, Number, Time, and Mode.

% TODO: Rotate 'Person' text.
\begin{center}
  Time is Present, Past, or Future.

  \textsc{to have}.

  Indicative Mode.

  Present Time.

  \begin{tabular}[h]{llll}
    & 1. I have, & We\\
    Person & 2. Thou hast,\footnotemark & Ye & have.\\
    & 3. He hath, or has;\footnotemark & They\\
  \end{tabular}

  Past Time.

  \begin{tabular}[h]{lll}
    1. I had, & We\\
    2. Thou hadst, & Ye & had.\\
    3. He had; & They\\
  \end{tabular}

  Future Time.

  \begin{tabular}[h]{lll}
    1. I shall, or will, & We & shall,\\
    2. Thou shalt, or wilt\footnotemark, have; & Ye & or will,\\
    3. He shall, or will, & They & have.\\
  \end{tabular}

  Imperative Mode.

  \begin{tabular}[h]{ll}
    1. Let me have, & Let us have,\\
    2. Have thou, & Have ye,\\
    or, Do thou have, & or, Do ye have,\\
    3. Let him have; & Let them have.\\
  \end{tabular}

  Subjunctive Mode.

  Present Time.

  \begin{tabular}[h]{llll}
    1. I & & We\\
    2. Thou & have; & Ye & have.\\
    3. He & & They\\
  \end{tabular}

  Infinitive Mode.

  Present, To have: Past, To have had.

  Participle.

  Present, Having: Perfect,\footnotemark Had:

  Past, Having had.
\end{center}

% TODO: Footnote from missing page/s is here; ignore for now?

\footnotetext{\tl{Thou}, in the Polite, and even in the Familiar Style,
  is disused, and the Plural \tl{You} is employed instead of it: we say,
  \tl{You have}; not, \tl{Thou hast}. Though in this case we apply
  \tl{You} to a single Person, yet the Verb too must agree with it in
  the Plural Number: it must necessarily be, \tl{You have}, not, \tl{You
    hast}. \tl{You was}, the Second Person Plural of the Pronoun placed
  in agreement with the first or Third Person Singular of the Verb, is
  an enormous Solecism: and yet Authors of the first rank have
  inadvertently fallen into it. ``Knowing that \tl{you was} my old
  master's good friend.'' Addison, Spect. No. 517. ``The account \tl{you
    was} pleased to send me.'' Bentley, Phileleuth. Lipf. Part II. See
  the Letter prefixed. ``Would to God \tl{you was} within her reach!''
  Bolingbroke to Swift, Letter 46. ``If \tl{you was} here.'' Ditto,
  Letter 47. ``I am just now as well, as when \tl{you was} here.'' Pope
  to Swift, P.\ S.\ to Letter 56. On the contrary the Solemn Style
  admits not of \tl{You} for a single Person. This hath led Mr. Pope
  into a great impropriety in the beginning of his Messiah:

  \begin{quote}
    O \tl{Thou} my voice inspire,\\
    Who \tl{touch'd} Isaiah's hallow'd lips with fire.
  \end{quote}

  The Solemnity of the Style would not admit of \tl{You} for \tl{Thou}
  in the Pronoun, nor the measure of the Verse \tl{touchedst}, or
  \tl{didst touch}, in the Verb; as it indispensably ought to be, in the
  one, or the other, of these two forms: \tl{You}, who \tl{touched}; or
  \tl{Thou}, who \tl{touchedst}, or \tl{didst touch}.

  \begin{aquote}{Pope's Iliad, x. 90.}
    What art \tl{thou}, speak, \tl{that} on designs unknown,\\
    While others sleep, thus \tl{range} the camp alone?
  \end{aquote}

  \begin{aquote}{Ib. xix. 319.}
    Accept these grateful tears; for thee they flow;\\
    For \tl{thee}, \tl{that} ever \tl{felt} another's woe.
  \end{aquote}

  \begin{aquote}{Dr. Arbuthnot, Dodsley's Poems, vol. i.}
    Faultless \tl{thou dropt} from his unerring skill.
  \end{aquote}

  Again:

  \begin{aquote}{Pope, Epitaph.}
    Just of \tl{thy} word, in every thought sincere;\\
    Who \tl{knew} no wish, but what the world might hear.
  \end{aquote}

  It ought to be \tl{your} in the first line, or \tl{knewest} in the
  second.

  In order to avoid this Grammatical Inconvenience, the two distinct
  forms of \tl{Thou} and \tl{You} are often used promiscuously by our
  modern Poets, in the same Poem, in the same Paragraph, and even in the
  same Sentence; very inelegantly and improperly:

  \begin{aquote}{Pope.}
    Now, now, I seize, I clasp \tl{thy} charms;\\
    And now \tl{you} burst, ah cruel! from my arms.
  \end{aquote}}
\footnotetext{\tl{Hath} properly belongs to the serious and solemn
  style; \tl{has}, to the familiar. The same may be observed of
  \tl{doth} and \tl{does}.

  \begin{aquote}{Waller.}
    But, confounded with thy art,\\
    Inquires her name, that \tl{has} his heart.
  \end{aquote}

  \begin{aquote}{Addison.}
    Th' unwearied Sun from day to day\\
    \tl{Does} his Creator's pow'r display.
  \end{aquote}

  The nature of the style, as well as the harmony of the verse, seems to
  require in these places \tl{hath} and \tl{doth}.}

\footnotetext{The Auxiliary Verb \tl{will} is always thus formed in the
  second and third Persons singular: but the Verb \tl{to will}, not
  being an Auxiliary, is formed regularly in those Persons: I \tl{will},
  Thou \tl{willest}, He \tl{willeth}, or \tl{wills}. ``Thou, that art
  the author and bestower of life, canst doubtless restore it also, if
  thou \tl{will'st}, and when thou \tl{will'st}: but whether thou
  \tl{will'st} (wilt) please to restore it, or not, that Thou alone
  knowest.'' Atterbury, Serm. I. 7.}

\footnotetext{This Participle represents the action as complete and
  finished; and, being subjoined to the Auxiliary \tl{to have},
  constitutes the perfect Time: I call it therefore the Perfect
  Participle. The same, subjoined to the Auxiliary \tl{to be},
  constitutes the Passive Verb; and in that state, or when used without
  the Auxiliary in a Passive sense, is called the Passive Participle.}

\begin{center}
  \textsc{to be}:

  Indicative Mode.

  Present Time.

  \begin{tabular}[h]{lll}
    1. I am, & We\\
    2. Thou art, & Ye & are.\\
    3. He is; & They\\
  \end{tabular}

  Or,

  \begin{tabular}[h]{lll}
    1. I be, & We\\
    2. Thou beest, & Ye & be.\\
    3. He is;\footnotemark & They\\
  \end{tabular}

  Past Time.

  \begin{tabular}[h]{lll}
    1. I was, & We\\
    2. Thou wast, & Ye & were.\\
    3. He was; & They\\
  \end{tabular}

  Future Time.

  \begin{tabular}[h]{llll}
    1. I shall, or will, & & We & shall,\\
    2. Thou shalt, or wilt, & be; & Ye & or will,\\
    3. He shall, or will, & & They & be.\\
  \end{tabular}

  Imperative Mode.

  \begin{tabular}[h]{ll}
    1. Let me be, & Let us be,\\
    2. Be thou, & Be ye,\\
    or, Do thou be, & or Do ye be,\\
    3. Let him be; & Let them be.\\
  \end{tabular}

  Subjunctive Mode.

  \begin{tabular}[h]{llll}
    1. I & & We\\
    2. Thou & be; & Ye & be.\\
    3. He & & They\\
  \end{tabular}

  Past Time.

  \begin{tabular}[h]{lll}
    1. I were, & We\\
    2. Thou wert,\footnotemark & Ye & were.\\
    3. He were; & They\\
  \end{tabular}

  Infinitive Mode.

  Present, To be: Past, To have been.

  Participle.

  Present, Being: Perfect, Been:

  Past, Having been.
\end{center}

\footnotetext{``I think it \tl{be} thine indeed: for thou liest in it.''
  Shakspeare, Hamlet. \tl{Be}, in the Singular Number of this Time and
  Mode, especially in the third Person, is obsolete; and is become
  somewhat antiquated in the Plural.}

\footnotetext{
  \begin{aquote}{Milton.}
    Before the sun,\\
    Before the heav'ns thou \tl{wert}.
  \end{aquote}

  \begin{aquote}{Dryden.}
    Remember what thou \tl{wert}.
  \end{aquote}

  \begin{aquote}{Addison.}
    I knew thou \tl{wert} not slow to hear.
  \end{aquote}

  \begin{aquote}{Prior.}
    Thou who of old \tl{wert} sent to Israel's court.
  \end{aquote}

  \begin{aquote}{Pope.}
    All thou thou \tl{wert}.
  \end{aquote}

  \begin{aquote}{Swift.}
    Thou, Stella, \tl{wert} no longer young,\\
    When first for thee my harp I string.
  \end{aquote}

  Shall we in deference to these great authorities allow \tl{wert} to be
  the same with \tl{wast}, and common to the Indicative and Subjunctive
  Mode? or rather abide by the practice of our best ancient authors; the
  propriety of the language, which requires, as far as may be, distinct
  forms for different Modes; and the analogy of formation in each Mode;
  I \tl{was}, Thou \tl{wast}; I \tl{were}, Thou \tl{wert}? all which
  conspire to make \tl{wert} peculiar to the Subjunctive Mode.}

\begin{center}
  The Verb Active is thus varied according to\\
  Person, Number, Time and Mode.

  Indicative Mode.

  Present Time.

  % TODO: Rotated text
  \begin{tabular}[h]{llll}
    & Sing. & Plur.\\
    & 1. I love, & We\\
    Person & 2. Thou lovest, & Ye & love.\\
    & 3. He loveth, or loves; & They\\
  \end{tabular}

  Past Time.

  \begin{tabular}[h]{lll}
    1. I loved, & We\\
    2. Thou lovedst, & Ye & loved.\\
    3. He loved; & They\\
  \end{tabular}

  Future Time.

  \begin{tabular}[h]{llll}
    1. I shall, or will, & & We & shall,\\
    2. Thou shalt, or wilt, & love; & Ye & or will,\\
    3. He shall, or will, & & They & love.\\
  \end{tabular}

  Imperative Mode.

  \begin{tabular}[h]{ll}
    1. Let me love, & Let us love,\footnotemark\\
    2. Love thou, & Love ye,\\
    or, Do thou love, & or, Do ye love,\\
    3. Let him love; & Let them love.\\
  \end{tabular}

  Subjunctive Mode.

  Present Time.

  \begin{tabular}[h]{llll}
    1. I & & We\\
    2. Thou & love; & Ye & love.\\
    3. He & & They\\
  \end{tabular}

  And,

  \begin{tabular}[h]{llll}
    1. I may & & We & may love;\\
    2. Thou mayest & love; & Ye & and\\
    3. He may & & They & have loved.\footnotemark\\
  \end{tabular}
\end{center}

\footnotetext{The other form of the First Person Plural of the
  Imperative, \tl{love we}, is grown obsolete.}

\footnotetext{Note, that the Imperfect and Perfect Time are here put
  together. And it is to be observed, that in the Subjunctive Mode, the
  event being spoken of under a condition, or supposition, or in the
  form of a wish, and therefore as doubtful and contingent, the Verb
  itself in the Present, and the Auxiliary both of the Present and Past
  Imperfect Times, often carry with them somewhat of a Future sense: as,
  ``If he come to-morrow, I may speak to him:''---``if he should, or
  would, come to-morrow, I might, would, could, or should, speak to
  him.'' Observe also, that the Auxiliaries \tl{should} and \tl{would}
  in the Imperfect Times are used to express the Present and Future as
  well as the Past; as, ``It \tl{is} my desire, that he \tl{should}, or
  \tl{would}, come \tl{now}, or \tl{to-morrow};'' as well as, ``It
  \tl{was} my desire, that he \tl{should}, or \tl{would}, come
  \tl{yesterday}.'' So that in this Mode the precise Time of the Verb is
  very much determined by the nature and drift of the Sentence.}

% TODO: Missing page/s
MISSING PAGES!

and distinction. They are also of frequent and almost necessary use in
Interrogative and Negative Sentences. They sometimes also supply the
place of another Verb, and make the repetition of it, in the same or a
subsequent sentence, unecessary: as,

\begin{aquote}{Shakspeare, Jul. C\ae{}s.}
  He \tl{loves} not plays\\
  As thou \tl{dost}, Anthony.
\end{aquote}

\tl{Let} does not only express permission; but praying, exhorting,
commanding. \tl{May} and \tl{might} express the possibility or liberty
of doing a thing; \tl{can} and \tl{could}, the power. \tl{Must} is
sometimes called in for a helper, and denotes necessity. \tl{Will}, in
the first Person singular and plural, promises or threatens; in the
second and third Persons, only foretels: \tl{shall}, on the contrary, in
the first Person, simply foretels; in the second and third Persons,
promises, commands, or threatens.\footnote{This distinction was not
  observed formerly as to the word \tl{shall}, which was used in the
  Second and Third Persons to express simply the Event. So likewise
  \tl{should} was used, where we now make use of \tl{would}. See the
  vulgar Translation of the Bible.} But this must be understood of
Explicative Sentences; for when the Sentence is Interrogative, just the
reverse for the most part takes place: Thus, ``I \tl{shall} go; you
\tl{will} go;'' express event only: but, ``\tl{will} you go?'' imports
intention; and ``\tl{shall} I go?'' refers to the will of another. But
again, ``he \tl{shall} go,'' and ``\tl{shall} he go,'' both imply will,
expressing or referring to a command. \tl{Would} primarily denotes
inclination of will; and \tl{should}, obligation: but they both vary
their import, and are often used to express simple event.

\tl{Do} and \tl{have} make the Present Time; \tl{did},
\tl{had},\footnote{It has been very rightly observed, that the Verb
  \tl{had}, in the common phrase, \tl{I had rather}, is not properly
  used, either as an Active or as an Auxiliary Verb; that, being in the
  Past time, it cannot in this case be properly expressive of time
  Present; and that it is by no means reducible to any Grammatical
  construction. In truth, it seems to have arisen from a mere mistake,
  in resolving the familiar and amiguous abbreviation, \tl{I'd rather},
  into \tl{I had rather}, instead of \tl{I would rather}; which latter
  is the regular, analogous, and proper expression. See two Grammatical
  Essays. London. 1768. Essay 1.} the Past; \tl{shall}, \tl{will}, the
Future: \tl{let} is employed in forming the Imperative Mode; \tl{may},
\tl{might}, \tl{could}, \tl{would}, \tl{should}, in forming the
Subjunctive. The Preposition \tl{to}, placed before the Verb, makes the
Infinitive Mode.\footnote{Bishop Wilkins gives the following elegant
  investigation of the Modes, in his \tl{Real Character}, Part. III.
  Chap. 5.

  ``To show in what manner the Subject is to be joined with his
  Predicate, the Copula between them is affected with a Particle; which,
  from the use of it, is called \tl{Modus}, the manner or \tl{Mode}.

  Now the Subject and Predicate may be joined together either
  \tl{Simply}, or with some kind of \tl{Limitation}; and accordingly
  these Modes are Primary, or Secondary.

  The Primary Modes are called by Grammarians Indicative, and
  Imperative.

  When the matter is declared to be so, or at least when it seems in the
  Speaker's power to have it be so, as the bare union of Subject and
  Predicate would import; then the Copula is nakedly expressed without
  any variation: and this manner of expressing it is called the
  Indicative Mode.

  When it is neither declared to be so, nor seems to be immediately in
  the Speaker's power to have it so; then he can do no more in words,
  but make out the expression of his will to him that hath the thing in
  his power: namely, to

  \begin{tabular}[h]{lllll}
    & Superior, & & Petition,\\
    his & Equal, & by & Persuasion, & And the\\
    & Inferior, & & Command,\\
  \end{tabular}

  manner of these affecting the Copula, (Be it so, or, let it be so), is
  called the Imperative Mode; of which there are these three varieties,
  very fit to be distinctly provided for. As for that other use of the
  Imperative Mode, when it signifies \tl{Permission}: this may be
  sufficiently expressed by the \tl{Secondary Mode} of \tl{Liberty}; You
  \tl{may} do it.

  The Secondary Modes are such, as, when the Copula is affected with any
  of them, make the Sentence to be (as Logicians call it) a \tl{Modal
    Proposition}.

  This happens, when the matter in discourse, namely, the being, or
  doing, or suffering of a thing, is considered, not \tl{simply by
    itself}, but \tl{gradually in its causes}; from which it proceeds
  either \tl{contingently}, or \tl{necessarily}.

  Then a thing seems to be left as \tl{Contingent}, when the Speaker
  expresses only the \tl{Possibility} of it, or his own \tl{Liberty} to
  it.

  1. The \tl{Possibility} of a thing depends upon the power of its
  cause; and may be expressed,

  \begin{tabular}[h]{llll}
    \multirow{2}{*}{when} & \tl{Absolute}, & \multirow{2}{*}{by the
                                             Particle} & \tl{Can};\\
    & \tl{Conditional}, & & \tl{Could}.\\
  \end{tabular}

  2. The \tl{Liberty} of a thing depends upon a freedom from all
  obstacles either within or without, and is usually expressed in our
  language,

  \begin{tabular}[h]{llll}
    \multirow{2}{*}{when} & \tl{Absolute}, & \multirow{2}{*}{by the
                                             Particle} & \tl{May};\\
    & \tl{Conditional}, & & \tl{Might}.\\
  \end{tabular}

  Then a thing seems to be of \tl{Necessity}, when the Speaker
  expresseth the resolution of his own \tl{Will}, or some other
  \tl{Obligation} upon him from without.

  3. The \tl{Inclination of the Will} is expressed,

  \begin{tabular}[h]{llll}
    \multirow{2}{*}{if} & \tl{Absolute}, & \multirow{2}{*}{by the
                                           Particle}. & \tl{Will};\\
    & \tl{Conditional}, & & \tl{Would}.\\
  \end{tabular}

  4. The Necessity of a thing from some \tl{external Obligation},
  whether \tl{Natural} or \tl{Moral}, which we call Duty, is expressed,

  \begin{tabular}[h]{llll}
    \multirow{2}{*}{if} & \tl{Absolute}, & \multirow{2}{*}{by the
                                           Particle} & \tl{Must, ought,
                                                       shall};\\
    & \tl{Conditional}, & & \tl{Must, ought, should}.
  \end{tabular}

  See also \textsc{Hermes}, Book I. Chap. viii.''} \tl{Have}, through
its several Modes and Times, is placed only before the Perfect
Participle; and \tl{be}, in like manner, before the Present and Passive
Participles: the rest only before the Verb, or another Auxiliary, in its
Primary form.

When an Auxiliary is joined to the Verb, the Auxiliary goes through all
the variations of Person and Number; and the Verb itself continues
invariably the same. When there are two or more Auxiliaries joined to
the Verb, the first of them only is varied according to Person and
Number. The Auxiliary \tl{must} admits of no variation.

The Passive Verb is only the Participle Passive, (which for the most
part is the same with the Indefinite Past Time Active, and always the
same with the Perfect Participle,) joined to the Auxiliary Verb \tl{to
  be}, through all its Variations: as, ``I \tl{am loved}; I \tl{was
  loved}; I \tl{have been loved}, I \tl{shall be loved};'' and so on,
through all the Persons, the Numbers, the Times, and the Modes.

The Neuter Verb is varied like the Active; but, having somewhat of the
Nature of the Passive, admits in many instances of the Passive form,
retaining still the Neuter signification; chiefly in such Verbs, as
signify some sort of motion, or change of place or condition: as, ``I
\tl{am come}; I \tl{was gone}; I \tl{am grown}; I \tl{was
  fallen}.\footnote{I doubt much of the propriety of the following
  examples: ``The rules of our holy religion, from which we \tl{are}
  infinitely \tl{swerved}.'' Tillotson, Vol. I. Serm. 27. ``The whole
  obligation of that law and covenant, which God made with the Jews,
  \tl{was} also \tl{ceased}.'' Ib. Vol. II. Serm. 52. ``Whose number
  \tl{was} now \tl{amounted} to three hundred.'' Swift, Contests and
  Dissensions, Chap. 3. ``This Mareschal, upon some discontent, \tl{was
    entered} into a conspiracy against his master.'' Addison,
  Freeholder, No. 31. ``At the end of a Campaign, when half the men
  \tl{are deserted} or killed.'' Addison, Tatler, No. 42. Neuter Verbs
  are sometimes employed very improperly as Actives: ``Go, \tl{flee
    thee} away into the land of Judah.'' Amos, vii. 12. ``I think it by
  no means a fit and decent thing to \tl{vie Charities}, and erect the
  reputation of one upon the ruins of another.'' Atterbury, Serm. I. 2.
  ``So many learned men, that have spent their whole time and pains to
  \tl{agree} the Sacred with the Profane Chronology.'' Sir William
  Temple, Works, Fol. Vol. I. p. 295.

  \begin{aquote}{Pope, Odyss. xiv. 447.}
    How would \tl{the Gods my} righteous \tl{toils succeed?}
  \end{aquote}

  \begin{aquote}{Ibid, xxi. 219.}
    ---If \tl{Jove this arm succeed}.
  \end{aquote}

  And Active Verbs are as improperly made Neuter; as, ``I must
  \tl{premise} with three circumstances.'' Swift, Q. Anne's Last
  Ministry, Chap. 2. ``Those what think to \tl{ingratiate} with him by
  calumniating me.'' Bentley, Dissert. on Phalaris, p. 519.}'' The Verb
\tl{am, was}, in this case precisely defines the Time of the action or
event, but does not change the nature of it: the Passive form still
expressing, not properly a Passion, but only a state or condition of
Being.
%%% Local Variables:
%%% mode: latex
%%% TeX-master: "../main"
%%% End: