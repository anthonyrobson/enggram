\chapter{Adverb.}

\textsc{Adverbs} are \tl{added to Verbs}, and to Adjectives, to denote
some modification or circumstance of an action, or quality: as, the
manner, order, time, place, distance, motion, relation, quantity,
quality, comparison, doubt, affirmation, negation, demonstration,
interrogation.

In English they admit of no Variation; except some few of them, which
have the degrees of Comparison: as,\footnote{The formation of Adverbs in
  general with the Comparative and Superlative Terminations seems to be
  improper; at least it is now become almost obsolete: as, ``Touching
  things which generally are received,---we are \tl{hardliest} able to
  bring such proof of their certainty, as may satisfy gainsayers.''
  Hooker, B. v. 2. ``Was the \tl{easilier} persuaded.'' Raleigh. ``That
  he may the \tl{stronglier} provide.'' Hobbes, Life of Thucyd. ``The
  things \tl{highliest} important to the growing age.'' Shaftesbury,
  Letter to Molesworth. ``The question would not be, who loved himself,
  and who not; but, who loved and served himself the \tl{rightest}, and
  after the truest manner.'' Id. Wit and Humor. It ought rather to be,
  \tl{most hardly}, \tl{more easily}, \tl{more strongly}, \tl{most
    highly}, \tl{most right}, or \tl{rightly}. But these Comparative
  Adverbs, however improper in prose, are sometimes allowable in Poetry.

  \begin{aquote}{Milton, P.\ L.\ vi. 731.}
    Scepter and pow'r, thy giving, I assume;\\
    And \tl{gladlier} shall resign.
  \end{aquote}} ``often, oftener, oftenest;'' ``soon, sooner, soonest;''
and those Irregulars, derived from Adjectives\footnote{See above, p.
  39.} in this respect likewise irregular; ``well, better, best;'' \&c.

An adverb is sometimes joined to another Adverb, to modify or qualify
its meaning; as, ``very much; much too little; very prudently.''
%%% Local Variables:
%%% mode: latex
%%% TeX-master: "../main"
%%% End:
