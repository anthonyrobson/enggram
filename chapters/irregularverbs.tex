\chapter{Irregular Verbs.}

\textsc{In} English both the Past Time Active and the Participle
Perfect, or Passive, are formed by adding to the Verb \tl{ed}; or \tl{d}
only, when the Verb ends in \tl{e}: as, ``\tl{turn, turned}; \tl{love,
  loved}.'' The Verbs that vary from this rule, in either or in both
cases, are esteemed Irregular.

The nature of our language, the Accent and Pronunciation of it, inclines
us to contract even all our Regular Verbs: thus \tl{loved, turned}, are
commonly pronounced in one syllable, \tl{lov'd, turn'd}: and the second
Person, which was originally in three syllables, \tl{lovedest,
  turnedest}, is become a disyllable, \tl{lovedst, turnedst}; for as we
generally throw the accent as far back as possible towards the first
part of the word, (in some even to the fourth syllable from the end,)
the stress being laid on the first syllables, the rest are pronounced in
a lower tone, more rapidly and indistinctly; and so are often either
wholly dropped, or blended into one another.

It sometimes happens also, that the word, which arises from a regular
change, does not sound easily or agreeably; sometimes by the rapidity of
our pronunciation the vowels are shortened or lost; and the consonants,
which are thrown together, do not easily coalesce with one another, and
are therefore changed into others of the same organ, or of a kindred
species. This occasions a further deviation from the regular form: thus
\tl{loveth, turneth}, are contracted into \tl{lov'th, turn'th}, and
these for easier pronunciation immediately become \tl{loves, turns}.

Verbs ending in \tl{ch, ck, p, x, ll, ss}, in the Past Time Active, and
the Participle Perfect or Passive, admit the change of \tl{ed} into
\tl{t}; as,\footnote{Some of these Contractions are harsh and
  disagreeable: and it were better, if they were avoided and disused:
  but they prevail in common discourse, and are admitted into Poetry;
  which latter indeed cannot well do without them.} \tl{snatcht, checkt,
snapt, mixt}, dropping also one of the double letters, \tl{dwelt, past};
for \tl{snatched, checked, snapped, mixed, dwelled, passed}: those that
end in \tl{l, m, n, p}, after a diphthong, moreover shorten the
diphthong, or change it into a single short vowel; as, \tl{dealt,
  dreamt, meant, felt, slept}, \&c.: all for the same reason; from the
quickness of the pronunciation, and because the \tl{d} after a short
vowel will not easily coalesce with the preceding consonant. Those that
end in \tl{ve} change also \tl{v} into \tl{f}; as, \tl{bereave, bereft};
\tl{leave, left}; because likewise \tl{v} after a short vowel will not
easily coalesce with \tl{t}.

All these, of which I have hitherto given examples, are considered not
as Irregular, but as Contracted only: in most of them the Entire as well
as the Contracted form is used; and the Entire form is generally to be
preferred to the Contracted.

The formation of Verbs in English, both Regular and Irregular, is
derived from the Saxon.

The Irregular Verbs in English are all Monosyllables, unless compounded;
and they are for the most part the same words which are Irregular Verbs
in the Saxon.

As all our Regular Verbs are subject to some kind of Contraction; so the
first Class of Irregulars is of those, that become so from the same
cause.

\section{Irregulars by Contraction.}

Some Verbs ending in \tl{d} or \tl{t} have the Present, the Past Time,
and the Participle Perfect and Passive, all alike, without any
variation: as, beat, burst,\footnote{These two have also \tl{beaten} and
  \tl{bursten} in the Participle; and in that form they belong to the
  Third Class of Irregulars.} cast,\footnote{Shakspeare uses the
  Participle in the Regular Form:

  \begin{aquote}{Hen. V.}
    And when the mind is quicken'd, out of doubt\\
    The organs, though defunct and dead before,\\
    Break up their drowsie grave, and newly move\\
    With \tl{casted} slough, and fresh celerity.
  \end{aquote}} cost, cut, heat,*\footnote{``He commanded, that they
  should heat the furnace one seven times more than it was wont to be
  \tl{heat}.'' Dan. iii. 19.

  The Verbs marked thus *, throughout the three Classes of Irregulars,
  have the Regular as well as the Irregular Form in use.
} hit, hurt, knit, let, lift,* light,*\footnote{This Verb in the Past
  Time and Participle is pronounced short, \tl{light} or \tl{lit}: but
  the Regular Form is preferable, and prevails most in writing.} put,
quit,* read,\footnote{This Verb in the Past Time and Participle is
  pronounced short; \tl{read, red, red}; like \tl{lead, led, led}; and
  perhaps ought to be written in this manner: our ancient writers spelt
  it \tl{redde}.} rent, rid, set, shed, shred, shut, slit,
split,\footnote{Shakspeare uses the Participle in the Regular Form:

  \begin{aquote}{Ant. and Cleop.}
    That self hand,\\
    Which writ his honor in the acts it did,\\
    Hath, with the courage which the heart did lend it,\\
    \tl{Splitted} the heart itself.
  \end{aquote}} spread, trust, wet.*

These are Contractions from \tl{beated, bursted, casted}, \&c; because
of the disagreeable sound of the syllable \tl{ed} after \tl{d} or
\tl{t}.\footnote{They follow the Saxon rule: ``Verbs which in the
  Infinitive end in \tl{dan} and \tl{tan},'' (that is, in English,
  \tl{d} and \tl{t}; for \tl{an} is only the Characteristic termination
  of the Saxon Infinitive;) ``in the Preterit and Participle Preterit
  commonly, for the sake of better sound, throw away the final \tl{ed};
  as \tl{beot, afed}, (both in the Preterit and Participle Preterit,)
  for \tl{beoted, afeded}; from \tl{beotan, afedan}.'' Hickes, Grammat.
  Saxon. cap. ix. So the same Verbs in English, \tl{beat, fed}, instead
  of \tl{beated, feeded}.}

Others not ending in \tl{d} or \tl{t} are formed by Contraction; have,
\tl{had}, for \tl{haved}; make, \tl{made} for \tl{maked}; flee,
\tl{fled}, for \tl{flee-ed}; shoe, \tl{shod}, for \tl{shoe-ed}.

The following, beside the Contraction, change also the Vowel; sell,
sold; tell, told; clothe, clad.*

Stand, stood; and dare, durst, (which in the Participle hath regularly
\tl{dared},) are directly from the Saxon, \tl{standan, stod; dyrnan,
  dorste}.

\section{Irregulars in \emph{ght}.}

The Irregulars of the Second Class end in \tl{ght}, both in the Past
Tense and Participle; and change the vowel or diphthong into \tl{au} or
\tl{ou}: they are taken from the Saxon, in which the termination is
\tl{hte}.

\begin{tabular}[h]{llll}
  & & \multicolumn{2}{c}{Saxon.}\\
  Bring, & brought: & Bringan, & brohte.\\
  Buy, & bought: & Bycgean, & bohte.\\
  Catch, & caught:\\
  Fight, & fought:\footnotemark & Feotan: & fuht.\\
\end{tabular}

\footnotetext{
  \begin{aquote}{Shakspeare, Hen. V.}
    As in this glorious and well-\tl{foughten} field\\
    We kept together in our chivalry.
  \end{aquote}

  \begin{aquote}{Milton, P. L. VI. 410.}
    On the \tl{foughten} field\\
    Michael, and his Angels, prevalent,\\
    Encamping, plac'd in guard their watches round.
  \end{aquote}}

\begin{tabular}[h]{llll}
  & & \multicolumn{2}{c}{Saxon.}\\
  Teach, & taught: & T\ae{}chan, & t\ae{}hte.\\
  Think, & thought: & Thencan, & thohte.\\
  Seek, & sought: & Secan, & sohte.\\
  Work, & wrought: & Weorcan, & wrohte.\\
\end{tabular}

\tl{Fraught} seems rather to be an Adjective than the Participle of the
Verb \tl{freight}, which has regularly \tl{freighted}. \tl{Raught} from
\tl{reach} is obsolete.

\section{Irregulars in \emph{en}.}

The Irregulars of the Third Class form the Past Time by changing the
vowel or diphthong of the Present; and the Participle Perfect and
Passive, by adding the termination \tl{en}; beside, for the most part,
the change of the vowel or diphthong. These also derive their formation
in both parts from the Saxon.

\begin{tabular}[h]{lll}
  Present. & Past. & Participle.\\
  \tl{a} changed into \tl{e}.\\
  Fall, & fell, & fallen.\\
\end{tabular}

This Participle seems not agreeable to the Analogy of derivation, which
obtains in this Class of Verbs.

\begin{tabular}[h]{lll}
  \tl{i} long into & \tl{o} & \tl{i} short.\\
  Abide, & abode.\\
  Climb, & clomb, & (climbed.)\\
  Drive, & drove, & driven.\\
  Ride, & rode, & ridden.\\
  Rise, & rose,\footnotemark & risen.\\
  Shine, & shone,* & (shined.)\\
  Shrive, & shrove, & shriven.\\
  Smite, & smote, & smitten.\\
  Stride, & strode, & stridden.\\
  Strive, & strove,* & striven.*\\
  Thrive, & throve,\footnotemark & thriven.\\
  Write,\footnotemark & wrote, & written.\\
\end{tabular}

\footnotetext{\tl{Rise}, with \tl{i} short, hath been improperly used as
the Past Time of this Verb: ``That form of the first or primigenial
earth, which \tl{rise} immediately out of Chaos, was not the same, nor
like to that of the present earth.'' Burnet, Theory of the Earth, B. I.
Chap. iv. ``If we hold fast to that scripture-conclusion, that all
mankind \tl{rise} from one head.'' Ibid. B. II. Chap. vii.}

\footnotetext{Mr. Pope has used the Regular form of the Past Time of
  this Verb:

  \begin{aquote}{Essay on Crit.}
    In the fat age of pleasure, wealth, and ease,\\
    Sprung the rank weed, and \tl{thriv'd} with large increase.
  \end{aquote}}

\footnotetext{This Verb is also formed like those of \tl{i} long into
  \tl{i} short; Write, writ, written: and by Contraction \tl{writ} in
  the Participle: but, I think, improperly.}

\begin{tabular}[h]{lll}
  \tl{i} long into \tl{u}, & & \tl{i} short.\\
  Strike, & struck, & stricken, or strucken.\\
  \tl{i} short into \tl{a}.\\
  Bid, & bade, & bidden.\\
  Give, & gave, & given.\\
  Sit,\footnotemark & sat, & sitten.\\
\end{tabular}

\footnotetext{Frequent mistakes are made in the formation of the
  Participle of this Verb. The analogy plainly requires \tl{sitten};
  which was formerly in use: ``The army having \tl{sitten} there so
  long.''---``Which was enough to make him stir, that would not have
  \tl{sitten} still, though Hannibal had been quiet.'' Raleigh. ``That
  no Parliament should be dissolved, till it had \tl{sitten} five
  months.'' Hobbes, Hist. of Civil Wars, p. 257. But it is now almost
  wholly disused, the form of the Past Time \tl{sat} having taken its
  place. ``The court \tl{was sat}, before Sir Roger came.'' Addison,
  Spect. No. 122. See also Tatler, No. 253, and 265. Dr. Middleton hath,
  with great propriety, restored the true Participle.---``To have
  \tl{sitten} on the heads of the Apostles: to have \tl{sitten} upon
  each of them.'' Works, Vol. II. p. 30. ``Blessed is the man,---that
  hath not \tl{sat} in the seat of the scornful.'' Psal. i. 1. The old
  Editions have \tl{sit}; which may be perhaps allowed, as a Contraction
  of \tl{sitten}. ``And when he was \tl{set}, his disciples came unto
  him,'' Mat. v. 1.---``who is \tl{set} on the right hand,''---``and is
  \tl{set} down at the right hand of the throne of God;'' Heb. viii. 1.
  \& xii. 2. (see also Mat. xxvii. 19. Luke, xxii. 55. John, xxiii. 12.
  Rev. iii. 21.) \tl{Set} can be no Part of the Verb \tl{to sit}. If it
  belong to the Verb \tl{to set}, the Translation in these passages is
  wrong: for to \tl{set} signifies \tl{to place}, but without any
  designation of the posture of the person placed; which is a
  circumstance of importance expressed by the original.}

\begin{tabular}[h]{lll}
  Spit, & spat, & spitten.\\
  \tl{i} short into \tl{u}.\\
  Dig, & dug,* & (digged.)\\
  \tl{ie} into \tl{ay}.\\
  Lie,\footnotemark & lay, & lien, or lain.\\
  \tl{o} into & \tl{e}.\\
  Hold, & held, & holden.\\
  \tl{o} into & \tl{i}.\\
  Do, & did, & done, i.e. doen.\\
  \tl{oo} into & \tl{o}.\\
  Choose & chose. & chosen.\\
  \tl{ow} into & \tl{ew}.\\
  Blow, & blew, & blown.\\
  Crow, & crew, & (crowed.)\\
  Grow, & grew, & grown.\\
  Know, & knew, & known.\\
  Throw, & threw, & thrown.\\
\end{tabular}

\footnotetext{This Neuter Verb is frequently confounded with the Verb
  Active \tl{to lay}, (that is, to \tl{put} or \tl{place};) which is
  Regular, and has in the Past Time and Participle \tl{layed} or
  \tl{laid}.

  \begin{aquote}{Pope, Iliad. xxiv. 622.}
    For him, through hostile camps I bent my way;\\
    For him, thus prostrate at thy feet I \tl{lay}:\\
    Large gifts proportion'd to thy wrath I bear.
  \end{aquote}

  Here \tl{lay} is evidently used for the present Time, instead of
  \tl{lie}.}

\begin{tabular}[h]{lll}
  \tl{y} into & \tl{ew}. & \tl{ow}.\\
  Fly,\footnotemark & flew, & flown.\footnotemark\\
\end{tabular}

\footnotetext{That is, as a bird, \tl{volare}; whereas \tl{to flee}
  signifies \tl{fugere}, as from an enemy. So in the Saxon and German,
  \tl{fleogan}, \tl{fliegen}, \tl{volare}: \tl{fleon}, \tl{flichen},
  \tl{fugere}. This seems to be the proper distinction between \tl{to
    fly}, and to \tl{flee}; which in the Present Times are very often
  confounded. Our Translation of the Bible is not quite free from this
  mistake. It hath \tl{flee} for \tl{volare}, in perhaps seven or eight
  places out of a great number; but never \tl{fly} for \tl{fugere}.}

\footnotetext{
  \begin{aquote}{Roscommon, Essay.}
    For rhyme in Greece or Rome was never known,\\
    Till by barbarian deluges \tl{o'erflown}.
  \end{aquote}

  ``Do not the Nile and Niger make yearly inundations in our days, as
  they have formerly done? and are not the countries so \tl{overflown}
  still situate between the tropicks.'' Bentley's Sermons.

  \begin{aquote}{Swift.}
    Thus oft by mariners are shown\\
    Earl Godwin's castles \tl{overflown}.
  \end{aquote}

  Here the Participle of the Irregular Verb, to \tl{fly}, is confounded
  with that of the Regular Verb, \tl{to flow}. It ought to be in all
  these places \tl{overflowed}.}

The following are Irregular only in the Participle; and that without
changing the vowel.

\begin{tabular}[h]{lll}
  Bake, & (baked,) & baken.*\\
  Fold, & (folded,) & folden.*\footnotemark\\
\end{tabular}

\footnotetext{``While they be \tl{folden} together as thorns.'' Nahum,
  i. 10.}

That all these had originally the termination \tl{en} in the Participle,
is plain from the following consideration. \tl{Drink} and \tl{bind}
still retain it; \tl{drunken}, \tl{bounden}; from the Saxon,
\tl{druncen}, \tl{bunden}: and the rest are manifestly of the same
analogy with these. \tl{Begonnen}, \tl{sonken}, and \tl{founden}, are
used by Chaucer: and some others of them appear in their proper shape in
the Saxon: \tl{scruncen}, \tl{spunnen}, \tl{sprungen}, \tl{stungen},
\tl{wunden}. As likewise in the German, which is only another offspring
of the Saxon: \tl{begunnen}, \tl{geklungen}, \tl{getrunken},
\tl{gesungen}, \tl{gesunken}, \tl{gespunnen}, \tl{gesprungen},
\tl{gestunken}, \tl{geschwummen}, \tl{geschwungen}.

The following seem to have lost the \tl{en} of the Participle in the
same manner:

\begin{tabular}[h]{lll}
  Hang,\footnotemark & hung,* & hung.*\\
  Shoot, & shot, & shot.\\
  Stick, & stuck, & stuck.\\
  Come, & came, & come.\\
  Run, & ran, & run.\\
  Win, & won, & won.\\
\end{tabular}

\footnotetext{This Verb, when Active, may perhaps be most properly used
  in the Regular form; when Neuter, in the Irregular. But in the Active
  sense of \tl{furnishing a room with draperies} the Irregular form
  prevails. The Vulgar Translation of the Bible uses only the Regular
  form.}

\tl{Hangen}, and \tl{scoten}, are the Saxon originals of the two first
Participles; the latter of which is likewise still in use in its first
form in one phrase: a \tl{shotten} herring. \tl{Stuck} seems to be a
contraction from \tl{sticken}, as \tl{struck} now in use for
\tl{strucken}. Chaucer hath \tl{comen} and \tl{wonnen}: \tl{becommen} is
even used by Lord Bacon.\footnote{Essay xxix.} And most of them still
subsist entire in the German: \tl{gehangen}, \tl{kommen}, \tl{gerunnen},
\tl{gewonnen.}

To this third Class belong the Defective Verbs, Be, been; and Go, gone;
i.e. goen.

From this Distribution and account of the Irregular Verbs, if it be
just, it appears, that originally there was no exception from the Rule.
That the Participle Preterit, or Passive, in English ends in \tl{d},
\tl{t}, or \tl{n}. The first form included all the Regular Verbs; and
those, which are become Irregular by Contraction, ending in \tl{t}. To
the second properly belonged only those which end in \tl{ght}, from the
Saxon Irregulars in \tl{hte}. To the third, those from the Saxon
Irregulars in \tl{en}; which have still, or had originally, the same
termination.

The same Rule affords a proper foundation for a division of all the
English Verbs into Three Conjugations; or Classes of Verbs,
distinguished one from another by a peculiar formation, in some
principal part of the Verbs belonging to each: of which Conjugations
respectively the three different Terminations of the Participle might be
the Characteristics. Such of the contracted Verbs, as have their
Participles now ending in \tl{t}, might perhaps be best reduced to the
first Conjugation, to which they naturally and originally belonged; and
they seem to be of a very different analogy from those in \tl{ght}. But
as the Verbs of the first Conjugation would so greatly exceed in number
those of both the others, which together make but about
117;\footnote{The whole number of Verbs in the English language, Regular
and Irregular, Simple and Compounded, taken together, is about 4300.
See, in Dr. Ward's Essays on the English Language, the Catalogue of
English Verbs. The whole number of Irregular Verbs, the Defective
included, is about 177.} and as those of the third Conjugation are so
various in their form, and incapable of being reduced to one plain rule;
it seems better in practice to consider the first in \tl{ed} as the only
Regular form, and the others as deviations from it; after the example of
the Saxon and German Grammarians.

To the Irregular Verbs are to be added the Defective; which are not only
for the most part Irregular, but are also wanting in some of their
parts. They are in general words of most frequent and vulgar use; in
which Custom is apt to get the better of Analogy. Such are the Auxiliary
Verbs; most of which are of this number. They are in use only in some of
their Times and Modes; and some of them are a Composition of Times of
several Defective Verbs having the same signification.

\begin{tabular}[h]{lll}
  Present. & Past. & Participle.\\
  Am, & was, & been.\\
  Can, & could.\\
  Go, & went, & gone.\\
  May, & might.\\
  Must.\\
  Quoth, & quoth.\\
  Shall, & should.\\
  Weet, wit, or wot; & wot.\\
  Will, & would.\\
  Wis, & wist.\\
\end{tabular}

There are not in English so many as a Hundred Verbs, (being only the
chief part, but not all, of the Irregulars of the Third Class,) which
have a distinct and different form for the Past Time Active and the
Participle Perfect or Passive. The general bent and turn of the language
is towards the other form; which makes the Past Time and the Participle
the same. This general inclination and tendency of the language seems to
have given occasion to the introducing of a very great Corruption: by
which the Form of the Past Time is confounded with that of the
Participle in these Verbs, few in proportion, which have them quite
different from one another. This confusion prevails greatly in common
discourse, and is too much authorized by the example of some of our best
Writers.\footnote{
  \begin{aquote}{Milton, P. L. x. 517.}
    He would \tl{have spoke}.
  \end{aquote}

  \begin{aquote}{P. L. i. 621.}
    Words \tl{interwove} with sighs found out their way.
  \end{aquote}

  \begin{aquote}{Eiconoclast. xvii.}
    Those kings and potentates who \tl{have strove}.
  \end{aquote}

  \begin{aquote}{Sam. Ag. ver. 1752.}
    And to his faithful servant \tl{hath} in place\\
    \tl{Bore} witness gloriously.
  \end{aquote}

  \begin{aquote}{Comus, ver. 195.}
    And envious darkness, ere they could return,\\
    \tl{Had stole} them from me.
  \end{aquote}

  Here it is observable, that the Author's MS.\ and the first Edition
  have it \tl{stolne}.

  \begin{aquote}{P. R. iii. 36.}
    And in triumph \tl{had rode}.
  \end{aquote}

  \begin{aquote}{P. R. i. 165.}
    I \tl{have chose}\\
    This perfect man.
  \end{aquote}

  \begin{aquote}{Dryden, Fables.}
    The fragrant brier \tl{was wove} between.
  \end{aquote}

  \begin{aquote}{Shakespear, As you like it.}
    I will scarce think you \tl{have swam} in a Gondola.
  \end{aquote}

  \begin{aquote}{Dryden, Poems, Vol. II. p. 172.}
    Then finish what you \tl{have began};\\
    But scribble faster, if you can.
  \end{aquote}

  \begin{aquote}{Pope's Odyss. xi. 555.}
    And now the years a numerous train \tl{have ran};\\
    The blooming boy is ripen'd into man.
  \end{aquote}

  ``Which I \tl{had} no sooner \tl{drank}, \tl{but} I found a pimple
  rising in my forehead.'' Addison, Tatler, No. 131.

  % TODO: missing page here
  ``\tl{Have sprang}.'' Atterbury, Serm. I. 4. ``\tl{had spake}''} Thus
it is said, \tl{He begun}, for \tl{he began}; \tl{he run}, for \tl{he
  ran}; \tl{he drunk}, for \tl{he drank}: the Participle

% TODO: missing page here
MISSING PAGE!

Vulgar Translation of the Bible, which is the best standard of our
language, is free from this corruption, except in a few instances; as
\tl{hid} is used for \tl{hidden}; \tl{held}, for \tl{holden},
frequently; \tl{bid}, for \tl{bidden}; \tl{begot}, for \tl{begotten},
once or twice: in which, and a few other like words, it may perhaps be
allowed as a Contraction. And in some of these, Custom has established
it beyond recovery: in the rest it seems wholly inexcusable. The
absurdity of it will be plainly perceived in the example of some of
these Verbs, which Custom has not yet so perverted. We should be
immediately shocked at \tl{I have knew}, \tl{I have saw}, \tl{I have
  gave}, \&c. but our ears are grown familiar with \tl{I have wrote},
\tl{I have drank}, \tl{I have bore}, \&c. which are altogether as
barbarous.

There are one or two small Irregularities to be noted, to which some
Verbs are subject in the formation of the Present Participle. The
Present Participle is formed by adding \tl{ing} to the Verb; as,
\tl{turn}, \tl{turning}, Verbs ending in \tl{e} omit the \tl{e} in the
Present Participle: as, \tl{love}, \tl{loving}. Verbs ending with a
single consonant preceded by a single Vowel, and, if of more than one
Syllable, having the accent on the last Syllable, double the Consonant
in the Present Participle, as well as in every Part of the Verb in which
a Syllable is added: as \tl{put}, \tl{putting}, \tl{putteth};
\tl{forget}, \tl{forgetting}, \tl{forgetteth}; \tl{abet}, \tl{abetting},
\tl{abetted}.\footnote{Some Verbs having the Accent on the last Syllable
but one, as \tl{worship}, \tl{counsel}, are represented in the like
manner, as doubling the last consonant in the formation of those parts
of the Verb in which a Syllable is added; as \tl{worshipping},
\tl{counselling}. But this I rather judge to be a fault in the spelling;
which neither Analogy nor Pronunciation justifies.}
%%% Local Variables:
%%% mode: latex
%%% TeX-master: "../main"
%%% End:
