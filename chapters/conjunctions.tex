\chapter{Conjunction.}

\textsc{The Conjunction} connects or \emph{joins together} Sentences; so
as, out of two, to make one Sentence.

Thus, ``You, \tl{and} I, \tl{and} Peter, rode to London,'' is one
Sentence, made up of these three by the Conjunction \tl{and} twice
employed; ``You rode to London; I rode to London; Peter rode to
London.'' Again, ``You \tl{and} I rode to London, \tl{but} Peter staid
at home,'' is one Sentence made up of three by the Conjunctions \tl{and}

% TODO: Missing pages
MISSING PAGES!

it is required to be in like case, number, gender, or person.

One word is said to govern another, when it causeth the other to be in
some Case, or Mode.

Sentences are either Simple, or Compounded.

A Simple Sentence hath in it but one Subject, and one Finite Verb; that
is, a Verb in the Indicative, Imperative, or Subjunctive Mode.

A Phrase is two or more words rightly put together, in order to make a
part of a Sentence; and sometimes making a whole Sentence.

The most common \textsc{Phrases}, used in simple Sentences, are the
following.

1st Phrase: The Substantive before a Verb Active, Passive, or Neuter;
when it is said, what thing \tl{is}, \tl{does}, or \tl{is done}: as, ``I
am;'' ``Thou writest;'' ``Thomas is loved:'' where \tl{I}, \tl{Thou},
\tl{Thomas}, are the Nominative\footnote{``He, \tl{whom} ye pretend
  reigns in heaven, is so far from protecting the miserable sons of men,
that he perpetually delights to blast the sweetest flowrets in the
Garden of Hope.'' Adventurer, No. 76. It ought to be \tl{who}, the
Nominative Case to \tl{reigns}; not \tl{whom}, as if it were the
Objective Case governed by \tl{pretend}. ``If you were here, you would
find three or four in the parlour after dinner, \tl{whom} you would say
passed their time agreeably.'' Locke, Letter to Molyneux.

\begin{aquote}{Dryden, Poems, Vol. II.\ p. 220.}
  Scotland and \tl{Thee} did each in other love.
\end{aquote}

\begin{aquote}{Shakspeare, 2 Henry VI.}
  We are alone; here's none, but \tl{Thee} and I.
\end{aquote}

It ought in both places to be \tl{Thou}; the Nominative Case to the Verb
expressed or understood.} Cases; and answer to the question, \tl{who},
or \tl{what}? as, ``Who is loved? Thomas.'' And the Verb agrees with the
Nominative Case in Number and Person;\footnote{
  \begin{aquote}{Dryden, Fables.}
    But \tl{Thou}, false Arcite, never \tl{shall} obtain\\
    Thy bad pretence.
  \end{aquote}

  It ought to be \tl{shalt}. The mistake seems to arise from the
  confounding of \tl{Thou} and \tl{You}, as equivalent in every respect;
  whereas one is Singular, the other Plural. See above, p. 46.

  \begin{aquote}{Cowley, on the Death of Hervey.}
    And wheresoe'er \tl{thou casts} thy view.
  \end{aquote}

  \begin{aquote}{Shakspeare, Jul. C\ae{}s.}
    There's (there \tl{are}) \tl{two} or \tl{three} of us have seen
    strange fights.
  \end{aquote}

  \begin{aquote}{Pope, P.\ S.\ to the Odyssey.}
    Great \tl{pains has} (have) been taken.
  \end{aquote}

  ``I have considered, \tl{what have} (hath) been said on both sides in
  this controversy.'' Tillotson, Vol. I.\ Serm. 27.

  ``One would think, there \tl{was} more \tl{Sophists} than one had a
  finger in this Volume of Letters.'' Bentley, Dissert. on Socrate's
  Epistles, Sect. IX.

  ``The \tl{number} of the names together \tl{were} about an hundred and
  twenty.'' Acts, i. 15. See also Job, xiv. 5.

  ``And Rebekah took goodly \tl{raiment} of her eldest son Esau which
  \tl{were} with her in the house, and put \tl{them} upon Jacob her
  youngest son.'' Cen. xxvii. 15.

  ``If the blood of bulls and of goats, and the \tl{ashes} of an heifer,
  sprinkling the unclean, \tl{santifieth} to the purifying of the
  flesh.'' Heb. ix. 13. See also Exod. ix. 8, 9, 10.} as, \tl{thou}
being the Second Person Singular, the Verb \tl{writest} is so too.

2nd Phrase: The Substantive after a Verb Neuter or Passive; when it is
said, that such a thing \tl{is}, or \tl{is made}, or \tl{thought}, or
\tl{called}, such \tl{another thing}; or, when the Substantive after the
Verb is spoken of the same thing or person with the Substantive before
the Verb: as, ``A calf becomes an ox;'' ``Plautus is accounted a Poet;''
``I am He.'' Here the latter Substantive is in the Nominative Case, as
well as the former; and the Verb is said to govern the Nominative Case:
or, the latter Substantive may be said to agree in Case with the former.

3rd Phrase: The Adjective after a Verb Neuter or Passive, in like
manner, as, ``Life \tl{is short}, and Art \tl{is long}.'' ``Exercise
\tl{is esteemed wholesome}.''

4th Phrase: The Substantive after a Verb Active, or Transitive: as when
one thing is said to \tl{act} upon, or \tl{do} something to, another:
as, ``to open a door;'' ``to build a house:'' ``Alexander conquered the
Persians.'' Here the thing acted upon is in the Objective
Case:\footnote{
  \begin{aquote}{Shakspeare, Merch. of Venice.}
    For \tl{who love} I so love?
  \end{aquote}

  \begin{aquote}{Id. Twelfth Night.}
    \tl{Whoe'er} I \tl{woo}, myself would be his wife.
  \end{aquote}

  \begin{aquote}{Id. Hen. VIII.}
    \tl{Whoever} the King \tl{favours},\\
    The Cardinal will find employment for,\\
    And far enough from court.
  \end{aquote}

  \begin{aquote}{Dryden, Juvenal, Sat. vi.}
    Tell who \tl{loves who}; what favours some partake,\\
    And who is jilted for another's sake.
  \end{aquote}

  ``Those, \tl{who} he \tl{thought} true to his party.'' Clarendon,
  Hist. Vol. I.\ p. 667, 8vo. ``\tl{Who} should I \tl{meet} the other
  night, but my old friend?'' Spect. No. 32. ``\tl{Who} should I
  \tl{see} in the lid of it, but the Doctor?'' Addison, Spect. No. 57.
  ``Laying the suspicion upon somebody, I know not \tl{who}, in the
  country.'' Swift, Apology prefixed to Tale of a Tub. In all these
  places it ought to be \tl{whom}.} as it appears plainly when it is
expressed by the Pronoun, which has a proper termination for that Case;
``Alexander conquered \tl{them};'' and the Verb is said to govern the
Objective Case.

5th Phrase: A Verb following another Verb, as, ``Boys love to play:''
where the latter Verb is in the Infinitive Mode.

6th Phrase: When one thing is said to belong to another: as, ``Milton's
poems:'' where the thing to which the other belongs is placed first, and
is in the Possessive Case; or else last, with the Preposition \tl{of}
before it: as, ``the poems of Milton.''

7th Phrase: When another Substantive is added to express and explain the
former more fully; as, ``Paul the Apostle;'' ``King George:'' where they
are both in the same case; and the latter is said to be put in
Apposition to the former.

8th Phrase: When the quality of the Substantive is expressed by adding
an Adjective to it: as, ``a wise man;'' ``a black horse.'' Participles
have the nature of Adjectives; as, ``a learned man;'' ``a loving
father.''

9th Phrase: An Adjective with a Verb in the Infinitive Mode following
it: as, ``worthy to die;'' ``fit to be trusted.''

10th Phrase: When a circumstance is added to a Verb, or to an Adjective,
by an Adverb: as, ``You read well;'' ``he is very prudent.''

11th Phrase: When a circumstance is added to a Verb, or an Adjective, by
a Substantive with a Preposition before it: as, ``I write for you;''
``he reads with care;'' ``studious of praise;'' ``ready for mischief.''

12th Phrase: When the same Quality in different Subjects is compared:
the Adjective in the Positive having after it the Conjunction \tl{as},
in the Comparative the Conjunction \tl{than}, and in the Superlative the
Preposition \tl{of}: as, ``white as snow;'' ``wiser than I;'' ``greatest
of all.''

The \textsc{Principal Parts} of a Simple Sentence are the Agent, the
Attribute, and the Object. The Agent is the thing chiefly spoken of; the
Attribute is the thing or action afformed or denied of it; and the
Object is the thing affted by such action.

In English the Nominative Case, denoting the Agent, usually goes before
the Verb, or Attribution; and the Objective Case, denoting the Object,
follows the Verb Active; and it is the order that determines the cases
in Nouns: as, ``Alexander conquered the Persians.'' But the Pronoun,
having a proper form for each of those cases, sometimes, when it is in
the Objective Case, is placed before the Verb; and, when it is in the
Nominative Case, follows the Object and Verb; as, ``Whom ye ignorantly
worship, \tl{him declare I} unto you.'' And the Nominative Case is
sometimes placed after a Verb Neuter: as, ``Upon thy right hand \tl{did
  stand the Queen}:'' ``On a sudden \tl{appeared the King}.'' And
always, when the Verb is accompanied with the Adverb \tl{there}: as,
``there \tl{was a man}.'' The reason of it is plain: the Neuter Verb not
admitting of an Objective Case after it, no ambiguity of Case can arise
from such a position of the Noun: and where no inconvenience attends it,
variety itself is pleasing.\footnote{
  \begin{aquote}{Atterbury, Sermons, I.\ 2.}
    It must then be meant of his sins who \tl{makes}, not of his who
    \tl{becomes}, \tl{the convert}.
  \end{aquote}

  \begin{aquote}{Pope, Essay on Man.}
    In him who \tl{is}, and him who \tl{finds}, \tl{a friend}.
  \end{aquote}

  ``Eye \tl{hath} not \tl{seen}, nor ear \tl{heard}, neither \tl{have
    entered} into the heart of man, \tl{the things}, which God hath
  prepared for them that love him.'' 1 Cor. ii. 9.

  There seems to be an impropriety in these sentence, in which the same
  Noun serves in a double capacity, performing at the same time the
  offices both of the Nominative and Objective Case.}

\tl{Who}, \tl{which}, \tl{what}, and the Relative \tl{that}, though in
the Objective Case, are always placed before the Verb; as are also their
Compounds, \tl{whoever}, \tl{whosoever}, \&c.: as, ``He \tl{whom} you
\tl{seek}.'' ``This is \tl{what}, or the thing \tl{which}, or \tl{that},
you \tl{want}.'' ``\tl{Whomsoever} you please \tl{to appoint}.''

When the Verb is a Passive
%%% Local Variables:
%%% mode: latex
%%% TeX-master: "../main"
%%% End:
