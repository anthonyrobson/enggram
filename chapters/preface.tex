\chapter{Preface}

\textsc{The} English language hath been much cultivated during the last
two hundred years. It hath been considerably polished and refined; its
bounds have been greatly enlarged; its energy, variety, richness, and
elegance, have been abundantly proved, by numberless trials, in verse
and in prose, upon all subjects, and in every kind of style: but,
whatever other improvements it may have received, it hath made no
advances in Grammatical Accuracy. \textsc{Hooker} is one of the earliest
writers, of considerable note, within the period above-mentioned: let
his writings be compared with the best of those of more modern date;
and, I believe, it will be found, that in correctness, propriety, and
purity of English style, he hath hardly been surpassed, or even equaled,
by any of his successors.

It is now about fifty years, since Doctor \textsc{Swift} made a public
remonstrance, addressed to the Earl of \textsc{Oxford}, then Lord
Treasurer, concerning the imperfect State of our Language; alledging in
particular, ``that in many instances it offended against every part of
Grammar.'' \textsc{Swift} must be allowed to have been a good judge of
this matter; to which he was himself very attentive, both in his own
writings, and in his remarks upon those of his friends: he is one of the
most correct, and perhaps the best, of our prose-writers. Indeed, the
justness of this complaint, as far as I can find, hath never been
questioned; and yet no effectual method hath hitherto been taken to
redress the grievance, which was the object of it.

But let us consider, how, and in what extent, we are to understand this
charge brought against the English Language: for the Author seems not to
have explained himself with sufficient clearness and precision on this
head. Does it mean, that the English Language, as it is spoken by the
politest part of the nation, and as it stands in the writings of our
most approved authors, often offends against every part of Grammar? Thus
far, I am afraid, the charge is true. Or does it further imply, that our
Language is in its nature irregular and capricious; not hitherto
subject, nor easily reducible, to a System of rules? In this respect, I
am persuaded, the charge is wholly without foundation.

The English Language is perhaps of all the present European Languages by
much the most simple in its form and construction. Of all the antient
Languages extant That is the most simple, which is undoubtedly the most
antient; but even that Language itself does not equal the English in
simplicity.

The words of the English Language are perhaps subject to fewer
variations from their original form, than those of any other. Its
Substantives have but one variation of Case; nor have they any
distinction of Gender, beside that which nature hath made. Its
Adjectives admit of no change at all, except that which expresses the
degrees of comparison. All the possible variations of the original form
of the Verb are not above six or seven; whereas in many Languages they
amount to some hundreds: and almost the whole business of Modes, Times,
and Voices, is managed with great ease by the assistance of eight or
nine commodious little Verbs, called from their use Auxiliaries. The
Construction of this Language is so easy and obvious, that our
Grammarians have thought it hardly worth while to give us any thing like
a regular and systematical Syntax. The English Grammar, which hath been
last presented to the public, and by the Person best qualified to have
given us a perfect one, comprises the whole Syntax in ten lines: for
this reason; ``because our Language has so little inflexion, that its
construction neither requires nor admits any rules.'' In truth, the
easier any subject is in its own nature, the harder is it to make it
more easy by explanation; and nothing is more unnecessary, and at the
same time commonly more difficult, than to give a formal demonstration
of a proposition almost self-evident.

It doth not then proceed from any peculiar irregularity or difficulty of
our Language, that the general practice both of speaking and writing it
is changeable with inaccuracy. It is not the Language, but the Practice,
that is in fault. The truth is, Grammar is very much neglected among us:
and it is not the difficulty of the Language, but on the contrary the
simplicity and facility of it, that occasions this neglect. Were the
Language less easy and simple, we should find ourselves under a
necessity of studying it with more care and attention. But as it is, we
take it for granted, that we have a competent knowledge and skill, and
are able to acquit ourselves properly, in our own native tongue: a
faculty, solely acquired by use, conducted by habit, and tried by the
ear, carries us on without reflexion; we meet with no rubs or
difficulties in our way, or we do not perceive them; we find ourselves
able to go on without rules, and we do not so much as suspect, that we
stand in need of them.

A Grammatical Study of our own Language makes no part of the ordinary
method of instruction, which we pass through in our childhood; and it is
very seldom that we apply ourselves to it afterward. Yet the want of it
will not be effectually supplied by any other advantages whatsoever.
Much practice in the polite world, and a general acquaintance with the
best authors, are good helps; but alone will hardly be sufficient: we
have writers, who have enjoyed these advantages in their full extent,
and yet cannot be recommended as models of an accurate style. Much less
then will what is commonly called Learning serve the purpose; that is, a
critical knowledge of antient Languages, and much reading of antient
authors: the greatest Critic and most able Grammarian of the last age,
when he came to apply his Learning and his Criticism to an English
Author, was frequently at a loss in matters of ordinary use and common
construction in his own \textsc{vernacular idiom}.

But perhaps the Notes subjoined to the following pages will furnish a
more convincing argument, than any thing that can be said here, both of
the truth of the charge of Inaccuracy brought against our Language, as
it subsists in Practice; and of the necessity of investigating the
Principles of it, and studying it Grammatically, if we would attain to a
due degree of skill in it. It is with reason expected of every person of
a liberal education, and it is indispensably required of every one who
undertakes to inform or entertain the public, that he should be able to
express himself with propriety and accuracy. It will evidently appear
from these Notes, that our best authors have committed gross mistakes,
for want of a due knowledge of English Grammar, or at least of a proper
attention to the rules of it. The examples there given are such as
occurred in reading, without any very curious or methodical examination:
and they might easily have been much increased in number by any one, who
had leisure or phlegm enough to go through a regular course of reading
with this particular view. However, I believe, they may be sufficient to
answer the purpose intended; to evince the necessity of the Study of
Grammar in our own Language; and to admonish those, who set up for
authors among us, that they would do well to consider this part of
Learning as an object not altogether beneath their regard.

The principal design of a Grammar of any Language is to teach us to
express ourselves with propriety in that Language; and to enable us to
judge of every phrase and form of construction, whether it be right or
not. The plain way of doing this is, to lay down rules, and to
illustrate them by examples. But, beside shewing what is right, the
matter may be further explained by pointing out what is wrong. I will
not take upon me to say, whether we have any Grammar, that sufficiently
instructs us by rule and example; but I am sure we have none, that, in
the manner here attempted, teaches us what is right by shewing what is
wrong; though this perhaps may prove the more useful and effectual
method of instruction.

Beside this principal Design of Grammar in our own Language, there is a
secondary use to which it may be applied, and which, I think, is not
attended to as it deserves; the facilitating of the acquisition of other
Languages, whether antient or modern. A good foundation in the General
Principles of Grammar is in the first place necessary for all those, who
are initiated in a learned education; and for all others likewise, who
shall have occasion to furnish themselves with the knowledge of modern
Languages. Universal Grammar cannot be taught abstractly: it must be
done with reference to some Language already known; in which the terms
are to be explained, and the rules exemplified. The learner is supposed
to be unacquainted with all, but his native tongue; and in what other,
consistently with reason and common sense, can you go about to explain
it to him? when he has a competent knowledge of the main principles of
Grammar in general, exemplified in his own Language; he then will apply
himself with great advantage to the study of any other. To enter at once
upon the Science of Grammar, and the study of a foreign Language, is to
encounter two difficulties together, each of which would be much
lessened by being taken separately and in its proper order. For these
plain reasons, a competent grammatical knowledge of our own language is
the true foundation, upon which all Literature, properly so called,
ought to be raised. If this method were adopted in our Schools; if
children were first taught the common principles of Grammar, by some
short and clear System of English Grammar, which happily by its
simplicity and facility is perhaps fitter than that of any other
Language for such a purpose; they would have some notion of what they
were going about, when they should enter into the Latin Grammar; and
would hardly be engaged so many years, as they now are, in that most
irksome and difficult part of Literature, with so much labor of the
memory, and with so little assistance of the understanding.

A design somewhat of this kind gave occasion to the following little
system, intended merely for a private and domestic use. The chief end of
it was to explain the general principles of Grammar, as clearly and
intelligibly as possible. In the definitions, therefore, easiness and
perspicuity have been sometimes preferred to logical exactness. The
common divisions have been complied with, as far as reason and truth
would permit. The known and received terms have been retained; except in
one or two instances, where others offered themselves, which seemed much
more significant. All disquisitions, which appeared to have more of
subtilty than of usefulness in them, have been avoided. In a word, it
was calculated for the use of the learner, even of the lowest class.
Those, who would enter more deeply into the Subject, will find it fully
and accurately handled, with the greatest acuteness of investigation,
perspicuity of explication, and elegance of method, in a treatise
intitled \textsc{Hermes}, by \textsc{James Harris}, Esq; the most
beautiful and perfect example of Analysis, that has been exhibited since
the days of \textsc{Aristotle}.

The author is greatly obliged to several Learned Gentlemen, who have
favored him with their remarks upon the first Edition, which was indeed
principally designed to procure their assistance, and to try the
judgement of the public. He hath endeavoured to weigh their
observations, without prejudice or partiality; and to make the best use
of the lights, which they have afforded him. He hath been enabled to
correct several mistakes; and encouraged carefully to revise the whole,
and to give it all the improvement which his present materials can
furnish. He hopes for the continuance of their favor, as he is sensible
there will still be abundant occasion for it. A system of this kind,
arising from the collection and arrangement of a multitude of minute
particulars, which often elude the most careful search, and sometimes
escape observation when they are most obvious, must always stand in need
of improvement. It is indeed the necessary condition of every work of
human art or science, small as well as great, to advance towards
perfection by slow degrees; by an approximation, which though it still
may carry it forward, yet will certainly never bring it to the point to
which it tends.
%%% Local Variables:
%%% mode: latex
%%% TeX-master: "../main"
%%% End:
